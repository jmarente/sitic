\section{Especificación de requisitos del sistema}

Para la creación de cualquier producto software, es necesario establecer las distintas condiciones y
necesidades que ha de satisfacer. Seguiremos un esquema que nos permita describir los requisitos de una
forma metódica y racional.

\subsection{Requisitos de interfaces externas}

En este apartado se describirá los requisitos de conexión del software y el hardware, así como la interfaz
de usuario.

De la conexión entre el software y el hardware se encarga la propial terminal del sistema, ya que al tratarse de
una herramienta de generación de contenidos, el usuario la ejecutará usando la terminal disponible que esté
instalada en el sistema que utilice.

A continuación, pasamos a definir la interfaz de la herramienta, es decir, todos los comandos que el usuario
podrá ejecutar:

\begin{itemize}
\item \textbf{Generación básica}: comando básico que permite al usuario generar un sitio
estático a partir de todos los recursos definidos.
\item \textbf{Modo watch}: comando que se queda a la espera de cualquier cambio realizado en los 
ficheros para generar de nuevo el sitio sin necesidad de ejecutar cada vez el comando básico.
\item \textbf{Modo servidor}: levanta un pequeño servidor en un puerto dado, permitiendo al usuario
navegar por el sitio web generado, sin necesidad de instalar ningún servidor y/o herramienta extra.
\item \textbf{Generación de traducciones}: comando que genera los ficheros de traducciones a partir de todas las 
templates de las que se compone el sitio.
\item \textbf{Compilación de traducciones}: generar los ficheros compilador que consume la herramienta para mostrar
las traducciones necesarias
\item \textbf{Scrapper de enlaces rotos}: herramienta que permita al usuario comprobar si existe algun enlace roto.
\end{itemize}


\subsection{Requisitos funcionales}

\subsection{Requisitos de rendimiento}

\subsection{Restricciones de diseño}

\subsection{Requisitos del sistema software}

\section{Modelo de casos de uso}

\subsection{Diagrama de los casos de uso}
\subsection{Descripción de los casos de uso}

\section{Modelo conceptual de datos}

\subsection{Diagrama de Entidad-Relación}

\section{Modelo de comportamiento del sistema}


\subsection{Diagramas de secuencia y contrato de las operaciones del sistema.}
