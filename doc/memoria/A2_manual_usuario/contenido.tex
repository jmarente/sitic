En este capítulo se explicarán en detalle las instrucciones de uso de la herramienta.

\section{Introducción a Sitic}

Sitic es un framework para sitios web de uso general. Técnicamente hablando, Sitic es un generador de sitios web estáticos.
A diferencia de otros sistemas que construyen dinámicamente una página cada vez que un visitante la solicita, Sitic crea el sitio
cuando creas su contenido. Dado que los sitios web se ven con mucha más frecuencia de lo que se edita,
Sitic está optimizado para la visualización del sitio web mientras proporciona una gran experiencia de escritura.

Los sitios construidos con Sitic son bastante más rápidos y muy seguros, que un sitio generado de forma dinámica.
Se pueden alojar en cualquier lugar, incluyendo Heroku, GitHub Pages, Firebase Hosting, Google Cloud Storage o Amazon
S3 entre otros. Los sitios de Sitic se ejecutan sin dependencias en tiempos de ejecución costosos como Ruby, Python
o PHP y sin dependencias en ninguna base de datos.

\subsection{En qué se diferencia con los generadores dinámicos}

Los generadores de sitios web generan contenidos en ficheros HTML. La mayoría son "generadores dinámicos".
Esto significa que el servidor HTTP (que es el programa que se ejecuta en su sitio web con el que el navegador del
usuario habla) ejecuta el generador para crear un nuevo fichero HTML cada vez que un usuario desea ver una página.

Crear la página de forma dinámica significa que la máquina que aloja el servidor HTTP tiene que tener suficiente
memoria y CPU para ejecutar el generador de forma eficaz durante todo el día. Si no, entonces el usuario tiene que
esperar a que la página se genere.

Nadie quiere que los usuarios esperen más de lo necesario, por lo que los generadores de sitios dinámicos programaron
sus sistemas para almacenar en caché los ficheros HTML. Cuando un fichero se almacena en caché, una copia de él se
almacena temporalmente en el equipo. Es mucho más rápido para que el servidor HTTP envíe esa copia la próxima vez que
se solicite la página que para generarla desde cero.

Sitic intenta llevar el almacenamiento en caché un paso más allá. Todos los ficheros HTML se representan en su máquina.
Puede revisar los ficheros antes de copiarlos en la máquina que aloja el servidor HTTP. Dado que los ficheros HTML
no se generan dinámicamente, decimos que Sitic es un "generador estático".

No tener que ejecutar la generación de HTML cada vez que se recibe una petición tiene varias ventajas. Entre ellas,
la más notable es el rendimiento, los servidores HTTP son muy buenos en el envío de ficheros. Tan bueno que puede
servir eficazmente el mismo número de páginas con una fracción de la memoria y la CPU necesaria para un sitio dinámico.

Sitic tiene dos componentes para ayudarle a construir y probar su sitio web. El que probablemente usará más a menudo es el
servidor HTTP incorporado. Cuando ejecuta el servidor, Sitic procesa todo su contenido en ficheros HTML y luego ejecuta
un servidor HTTP en su máquina para que pueda ver cómo son las páginas.

El segundo componente se utiliza cuando esté listo para publicar su sitio web en el equipo que ejecuta su sitio web.
Ejecutar Sitic sin ninguna acción reconstruirá su sitio web completo utilizando la configuración \texttt{base\_url} del fichero de
configuración de su sitio. Eso es necesario para que sus enlaces de página funcionen correctamente con la mayoría
de las empresas de alojamiento.

\subsection{Características principales}

En términos técnicos, Sitic toma un directorio fuente de ficheros y plantillas y los usa como entrada para crear un sitio web completo.

Sitic cuenta con las siguientes características:

\paragraph{General}

\begin{itemize}
\item Tiempos de generación rápidos
\item Fácil instalación
\item Aloje su sitio en cualquier lugar
\end{itemize}

\paragraph{Organización}

\begin{itemize}
\item Organización sencilla
\item Soporte para secciones
\item URL personalizables
\item Soporte para taxonomías configurables que incluyen categorías y etiquetas.
\item Capacidad de clasificar el contenido como usted desea
\item Generación automática de tabla de contenido
\item Creación dinámica de menú
\item Soporte de URLs legibles
\end{itemize}

\paragraph{Contenido}

\begin{itemize}
\item Soporte nativo para contenido escrito en Markdown, Textile y Reestructured text
\item Soporte internacionalización
\item Soporte para los metadatos TOML y YAML en los contenidos
\item Páginas completamente personalizables
\end{itemize}

%%%%%%%%%%%%%%%%%%%%%%%%%%%%%%%%%%%%%%%%%%%%%%%%%%%%%%
%%%%%%%%%%%%%%%%%%%%%%%%%%%%%%%%%%%%%%%%%%%%%%%%%%%%%%

\section{Estructura de ficheros}

Sitic toma un directorio y lo usa como entrada para crear una página web completa.

El nivel más alto del directorio principal tendrá los siguientes elementos:

\begin{verbatim}
+--content/
+--data/
+--locales/
+--templates/
+--static/
+--sitic.yml
\end{verbatim}

El proposito para cada fichero/directorio se decribe a continuación:

\begin{itemize}
    \item \textbf{content}: Aquí es donde se almacenan los contenidos de la web, se crearán
        sub-directorios para crear las distintas secciones de la web. Supongamos, que nuestra web
        tiene cuatro secciones: \texttt{blog}, \texttt{news}, \texttt{about} y \texttt{contact},
        entonces será necesario crear una carpeta para cada una de ellas.
    \item \textbf{data}: Este directorio contiene distintos ficheros de configuración que pueden
        ser usado mientras se genera la web. El contenido de estos ficheros puede estar en format
        YAML, JSON o TOML.
    \item \textbf{locales}: Ficheros con las traducciones de las cadenas usadas en las templates.
    \item \textbf{templates}: Los contenidos dentro de este directorio especifican como se covnertirán
        los contenidos en una web estática.
    \item \textbf{static}: Directorio usado almacenar todos los contenidos estáticos que la web
        necesitará como imñagenes, CSS, Javascript u otro tipo de contenido estático.
    \item \textbf{sitic.yml}: Todo proyecto hecho con sitic debe de tener un fichero
        de configuración en la raíz del proyecto. Este debe de tener el nombre \texttt{sitic.yml},
        usando el formato YAML~\cite{yaml}. Esta configuración se aplica a todo el siti completo,
        que incluye la \texttt{base\_url} y \texttt{title} de la página web.
\end{itemize}

\subsection{Ejemplo}
Un ejemplo de completo tendría el siguiente aspecto:

\begin{verbatim}
+-- content
|   +-- about.md
|   +-- blog
|       +-- post1.md
|       +-- post2.md
|       +-- post3.textile
|       +-- post4.rst
+-- locales
+-- dara
+-- sitic.yml
+-- static
|   +-- css
|   +-- fonts
|   +-- images
|   +-- js
+-- templates
    +-- base.html
    +-- default
    |   +-- list.html
    |   +-- page.html
    +-- section
        +-- about.html
\end{verbatim}

%%%%%%%%%%%%%%%%%%%%%%%%%%%%%%%%%%%%%%%%%%%%%%%%%%%%%%
%%%%%%%%%%%%%%%%%%%%%%%%%%%%%%%%%%%%%%%%%%%%%%%%%%%%%%

\section{Configuración}

La estructura de directorios de un sitio web de Sitic, o más exactamente de los ficheros de origen
que contienen su contenido y plantillas, proporciona la mayor parte de la información de configuración
que Sitic necesita. Por lo tanto, en esencia, muchos sitios web realmente no necesitan un fichero de
configuración. Esto se debe a que Sitic está diseñado para reconocer ciertos patrones de uso típicos
(y los espera por defecto).

Sin embargo, Sitic busca un fichero de configuración con un nombre en particular en la raíz del directorio
fuente de su sitio web. El fichero que busca es \texttt{./sitic.yaml}.

En este fichero de configuración para su sitio web, puede incluir instrucciones precisas a Sitic sobre cómo
debe procesar su sitio, así como definir sus menús y establecer varios otros parámetros de todo el sitio.

\subsection{Ejemplos}

A continuación se muestra el fichero de configuración básico:

\begin{yamlcode}
base_url: "http://sitic.example.com/"
\end{yamlcode}

Seguidamente podemos un fichero un poco más completo con distintos elementos:

\begin{yamlcode}
base_url: "www.sitic.net"
paginable: "1"
main_language: "en"

lazy_menu: "main"

sitemap:
    change_frequency: "monthly"
    priority: 0.5

menus:
    main:
        - title: "title test"
          url: "/test-url"
          id: "test1"
        - title: "title test2"
          url: "/test-url2"
          id: "test2-custom"
          parent: "test1"
    footer:
        - title: "title footer"
          url: "/test-url-footer"
          id: "test-footer"


languages:
    en:
    es:
        menus:
            footer:
                - title: "title footer"
                  url: "/test-url-footer"
                  id: "test-footer"
\end{yamlcode}

En este último ejemplo se pueden ver distintas configuraciones, como la configuración de menús o
idiomas, de las que hablaremos mas en detalle en las siguientes secciones.

% TODO: List all configurable variables in the config file
% \subsection{Variables configurables}

%%%%%%%%%%%%%%%%%%%%%%%%%%%%%%%%%%%%%%%%%%%%%%%%%%%%%%
%%%%%%%%%%%%%%%%%%%%%%%%%%%%%%%%%%%%%%%%%%%%%%%%%%%%%%

\section{Contenidos}

\subsection{Organización}

El contenido debe estar organizado de la misma manera que se pretende que aparezca en la web.
Sin ningún tipo de configuración adicional, el siguiente ejemplo funcionará:

\begin{verbatim}
content/
+-- about.md            // http://example.com/about
+-- blog                // http://example.com/blog/
    +-- post1.md        // http://example.com/blog/post1
    +-- post2.md        // http://example.com/blog/post2
    +-- post3.textile   // http://example.com/blog/post3
    +-- post3.rst       // http://example.com/blog/post4
\end{verbatim}

Como se puede apreciar, Sitic usará la ruta definida en en el disco a partir del directorio \texttt{content},
para definir la url del contenido, esto se puede sobreescribir usando el atributo \texttt{url} en los metadatos
de los contenidos.

Se puede usar el nivel de anidaciones que se desee, Sitic tomará siempre como \texttt{sección} los
directotios que se encuentren en el primer nivel.

\subsection{Formatos soportados}

Sitic soporta los siguientes formatos para los contenidos:

\begin{itemize}
    \item \textbf{Markdown}: Se identificarán aquellos ficheros con la extensión \texttt{.md}.
    \item \textbf{reStructuredText}: ficheros con la textensión \texttt{.textile}
    \item \textbf{Textile}: ficheros con cualquier de las extensiones \texttt{.txt} \texttt{.rst} \texttt{.rest} \texttt{.restx}
\end{itemize}

\subsection{Metadatos}

Sitic usa ficheros con encabezados para definir los metadatos, comúnmente llamados \texttt{front matter}.
Sitic usa la organización proporcionada para su contenido para minimizar cualquier
configuración adicional, aunque se puede sobrescribir en los metadatos.

Estos metadatos se identificarán a partir de unos caracteres delimitadores, los formatos soportados son:

\begin{itemize}
    \item \textbf{TOML}: identificado por +++.
    \item \textbf{YAML}: identificado por ---.
\end{itemize}

\textbf{Ejemplo de TOML}:

\begin{verbatim}
+++
title='Post 2'
description='Post 2 description'
tags=['tag1', 'tag2', 'tag3']
+++

El contenido va aquí
\end{verbatim}

\textbf{Ejemplo de YAML}:

\begin{verbatim}
---
title: 'Post 2'
description: 'Post 2 description'
tags: ['tag1', 'tag2', 'tag3']
---

El contenido va aquí
\end{verbatim}

% TODO
\subsubsection{Variables}

\subsection{Secciones}

Sitic cree que usted organiza su contenido con un propósito. La estructura usada
para organizar el contenido de origen se utiliza para organizar el sitio generado. Siguiendo
este patrón Sitic utiliza el nivel superior de su organización de contenido como la Sección.

Sitic creará automáticamente páginas para cada raíz de sección que enumere todo el contenido
de esa sección. Consulte la sección dedicada a las plantillas para obtener detalles sobre 
la personalización de la forma en que aparecen.

Las páginas de sección también pueden tener un fichero de contenido con metadatos.

\subsection{Taxonomías}

%%%%%%%%%%%%%%%%%%%%%%%%%%%%%%%%%%%%%%%%%%%%%%%%%%%%%%
%%%%%%%%%%%%%%%%%%%%%%%%%%%%%%%%%%%%%%%%%%%%%%%%%%%%%%

\section{Plantillas}

\subsection{Ruta de plantilla en función del tipo de contenido}

%%%%%%%%%%%%%%%%%%%%%%%%%%%%%%%%%%%%%%%%%%%%%%%%%%%%%%
%%%%%%%%%%%%%%%%%%%%%%%%%%%%%%%%%%%%%%%%%%%%%%%%%%%%%%

\section{Internacionalización}

\subsection{Configuración}
\subsection{Contenido}
\subsection{Plantillas}
\subsection{Generación y gestión}

%%%%%%%%%%%%%%%%%%%%%%%%%%%%%%%%%%%%%%%%%%%%%%%%%%%%%%
%%%%%%%%%%%%%%%%%%%%%%%%%%%%%%%%%%%%%%%%%%%%%%%%%%%%%%

\section{Comandos}

%%%%%%%%%%%%%%%%%%%%%%%%%%%%%%%%%%%%%%%%%%%%%%%%%%%%%%
%%%%%%%%%%%%%%%%%%%%%%%%%%%%%%%%%%%%%%%%%%%%%%%%%%%%%%

% TODO
% \section{Otras herramientas}
