En este capítulo se explicarán en detalle las instrucciones de uso de la herramienta.

\section{Introducción a Sitic}

Sitic es un \textit{framework} para hacer sitios web de uso general. Técnicamente hablando, Sitic es un generador de sitios web estáticos.
A diferencia de otros sistemas que genera dinámicamente una página cada vez que un visitante la solicita, Sitic crea el sitio
un única vez, 
cuando creas su contenido. Dado que los sitios web se ven con mucha más frecuencia de lo que se edita.

Los sitios construidos con Sitic son bastante más rápidos y seguros que un sitio generado de forma dinámica.
Se pueden alojar en cualquier lugar, incluyendo GitHub Pages, Google Cloud Storage o Amazon
S3 entre otros. Los sitios de Sitic se ejecutan sin depender de tiempos de ejecución costosos como Ruby, Python
o PHP y sin dependencia de ninguna base de datos.

\subsection{En qué se diferencia con los generadores dinámicos}

Los generadores de sitios web generan contenidos en ficheros HTML. La mayoría son "generadores dinámicos".
Esto significa que el servidor HTTP (que es el programa que se ejecuta en su sitio web con el que el navegador del
usuario habla) ejecuta el generador para crear un nuevo fichero HTML cada vez que un usuario desea ver una página.

Crear la página de forma dinámica significa que la máquina que aloja el servidor HTTP tiene que tener suficiente
memoria y CPU para ejecutar el generador de forma eficaz durante todo el día. Si no, entonces el usuario tiene que
esperar a que la página se genere.

Nadie quiere que los usuarios esperen más de lo necesario, por lo que los generadores de sitios dinámicos programaron
sus sistemas para almacenar en caché los ficheros HTML. Cuando un fichero se almacena en caché, una copia se
almacena temporalmente en el equipo. Es mucho más rápido que el servidor envíe esa copia la próxima vez que
se solicite la página en lugar generarla desde cero.

Sitic intenta llevar el almacenamiento en caché un paso más allá. Todos los ficheros HTML se representan en su máquina.
Puede revisar los ficheros antes de copiarlos en la máquina que aloja el servidor HTTP. Dado que los ficheros HTML
no se generan dinámicamente, decimos que Sitic es un "generador estático".

No tener que ejecutar la generación de HTML cada vez que se recibe una petición tiene varias ventajas. Entre ellas,
la más notable es el rendimiento, los servidores HTTP son muy buenos en el envío de ficheros. Tan bueno que puede
servir eficazmente el mismo número de páginas con una fracción de memoria y CPU necesaria para un sitio dinámico.

Sitic tiene dos componentes para ayudarle a construir y probar su sitio web. El que probablemente usará más a menudo es el
servidor HTTP incorporado. Cuando ejecuta el servidor, Sitic procesa todo su contenido en ficheros HTML y luego ejecuta
un servidor HTTP en su máquina para que pueda ver cómo son las páginas.

El segundo componente se utiliza una vez que el sitio esté listo para ser publicado.
Ejecutar Sitic sin ninguna acción reconstruirá su sitio web completo utilizando la configuración \texttt{base\_url} 
del fichero de configuración de su sitio. Eso es necesario para que sus enlaces de página funcionen correctamente 
con la mayoría de las empresas de alojamiento.

\subsection{Características principales}

En términos técnicos, Sitic toma un directorio fuente de ficheros 
y plantillas y los usa como entrada para crear un sitio web completo.

Sitic cuenta con las siguientes características:

\paragraph{General}

\begin{itemize}
\item Tiempos de generación rápidos
\item Fácil instalación
\item Posibilidad de alojar su sitio en cualquier lugar
\end{itemize}

\paragraph{Organización}

\begin{itemize}
\item Organización sencilla
\item Soporte para secciones
\item URL personalizables
\item Soporte para taxonomías configurables que incluyen categorías y etiquetas.
\item Capacidad de clasificar el contenido como usted desea
\item Generación automática de tabla de contenido
\item Creación dinámica de menú
\item Soporte de URLs legibles
\end{itemize}

\paragraph{Contenido}

\begin{itemize}
\item Soporte nativo para contenido escrito en Markdown, Textile y Reestructured text
\item Soporte internacionalización
\item Soporte para los metadatos TOML y YAML en los contenidos
\item Páginas completamente personalizables
\end{itemize}

%%%%%%%%%%%%%%%%%%%%%%%%%%%%%%%%%%%%%%%%%%%%%%%%%%%%%%
%%%%%%%%%%%%%%%%%%%%%%%%%%%%%%%%%%%%%%%%%%%%%%%%%%%%%%

\section{Estructura de ficheros}

Sitic toma un directorio y lo usa como entrada para crear una página web completa.

El nivel más alto del directorio principal tendrá los siguientes elementos:

\begin{verbatim}
    +--content/
    +--data/
    +--locales/
    +--templates/
    +--static/
    +--sitic.yml
\end{verbatim}

El proposito para cada fichero/directorio se decribe a continuación:

\begin{itemize}
    \item \textbf{content}: Aquí es donde se almacenan los contenidos de la web, se crearán
        sub-directorios para crear las distintas secciones de la web. Supongamos, que nuestra web
        tiene cuatro secciones: \texttt{blog}, \texttt{news}, \texttt{about} y \texttt{contact},
        entonces será necesario crear una carpeta para cada una de ellas.
    \item \textbf{data}: Este directorio contiene distintos ficheros de configuración que pueden
        ser usado mientras se genera la web. El contenido de estos ficheros puede estar en format
        YAML, JSON o TOML.
    \item \textbf{locales}: Ficheros con las traducciones de las cadenas usadas en las plantillas.
    \item \textbf{templates}: Los contenidos dentro de este directorio especifican como se convertirán
        los contenidos en una web estática.
    \item \textbf{static}: Directorio usado almacenar todos los contenidos estáticos que la web
        necesitará como imágenes, CSS, Javascript u otro tipo de contenido estático.
    \item \textbf{sitic.yml}: Todo proyecto hecho con Sitic debe de tener un fichero
        de configuración en la raíz del proyecto. Este debe de tener el nombre \texttt{sitic.yml},
        usando el formato YAML~\cite{yaml}. Esta configuración se aplica a todo el siti completo,
        que incluye la \texttt{base\_url} y \texttt{title} de la página web.
\end{itemize}

\subsection{Ejemplo}
Un ejemplo de completo tendría el siguiente aspecto:

\begin{verbatim}
    +-- content
    |   +-- about.md
    |   +-- blog
    |       +-- post1.md
    |       +-- post2.md
    |       +-- post3.textile
    |       +-- post4.rst
    +-- locales
    +-- dara
    +-- sitic.yml
    +-- static
    |   +-- css
    |   +-- fonts
    |   +-- images
    |   +-- js
    +-- templates
        +-- base.html
        +-- default
        |   +-- list.html
        |   +-- page.html
        +-- section
            +-- about.html
\end{verbatim}

%%%%%%%%%%%%%%%%%%%%%%%%%%%%%%%%%%%%%%%%%%%%%%%%%%%%%%
%%%%%%%%%%%%%%%%%%%%%%%%%%%%%%%%%%%%%%%%%%%%%%%%%%%%%%

\section{Configuración}

La estructura de directorios de un sitio web de Sitic, o más exactamente de los ficheros de origen
que contienen su contenido y plantillas, proporciona la mayor parte de la información de configuración
que Sitic necesita. Por lo tanto, en esencia, muchos sitios web realmente no necesitan un fichero de
configuración. Esto se debe a que Sitic está diseñado para reconocer ciertos patrones de uso típicos
(y los espera por defecto).

Sin embargo, Sitic busca un fichero de configuración con un nombre en particular en la raíz del directorio
de su sitio web. El fichero en concreto debe tener el nombre \texttt{./sitic.yaml}.

En este fichero de configuración puede incluir las instrucciones necesarias de cómo se
debe procesar su sitio, así como definir sus menús y establecer otras opciones.

\subsection{Ejemplos}

A continuación se muestra el fichero de configuración básico:

\begin{yamlcode}
    base_url: "http://Sitic.example.com/"
\end{yamlcode}

Seguidamente podemos un fichero un poco más completo con distintos elementos:

\begin{yamlcode}
    base_url: "www.Sitic.net"
    paginable: "1"
    main_language: "en"

    lazy_menu: "main"

    sitemap:
        change_frequency: "monthly"
        priority: 0.5

    menus:
        main:
            - title: "title test"
              url: "/test-url"
              id: "test1"
            - title: "title test2"
              url: "/test-url2"
              id: "test2-custom"
              parent: "test1"
        footer:
            - title: "title footer"
              url: "/test-url-footer"
              id: "test-footer"


    languages:
        en:
        es:
            menus:
                footer:
                    - title: "title footer"
                      url: "/test-url-footer"
                      id: "test-footer"
\end{yamlcode}

En este último ejemplo se pueden ver distintas configuraciones, como la configuración de menús o
idiomas, de las que hablaremos mas en detalle en las siguientes secciones.

% TODO: List all configurable variables in the config file
% \subsection{Variables configurables}

%%%%%%%%%%%%%%%%%%%%%%%%%%%%%%%%%%%%%%%%%%%%%%%%%%%%%%
%%%%%%%%%%%%%%%%%%%%%%%%%%%%%%%%%%%%%%%%%%%%%%%%%%%%%%

\section{Contenidos}

\subsection{Organización}

El contenido debe estar organizado de la misma manera que se pretende que aparezca en la web.
Sin ningún tipo de configuración adicional, a countinuacón se puede apreciar un ejemplo sencillo:

\begin{verbatim}
    content/
    +-- about.md            // http://example.com/about
    +-- blog                // http://example.com/blog/
        +-- post1.md        // http://example.com/blog/post1
        +-- post2.md        // http://example.com/blog/post2
        +-- post3.textile   // http://example.com/blog/post3
        +-- post3.rst       // http://example.com/blog/post4
\end{verbatim}

Como se puede apreciar, Sitic usará la ruta definida en en el disco a partir del directorio \texttt{content},
para definir la url del contenido, esto se puede sobreescribir usando el atributo \texttt{url} en los metadatos
de los contenidos.

Se puede usar el nivel de directorios que se desee, Sitic tomará siempre como \texttt{sección} los
directorios que se encuentren en el primer nivel.

\subsection{Formatos soportados}

Sitic soporta los siguientes formatos para los contenidos:

\begin{itemize}
    \item \textbf{Markdown}: Se identificarán aquellos ficheros con la extensión \texttt{.md}.
    \item \textbf{reStructuredText}: ficheros con la extensión \texttt{.textile}
    \item \textbf{Textile}: ficheros con cualquier de las extensiones \texttt{.txt} \texttt{.rst} \texttt{.rest} \texttt{.restx}
\end{itemize}

\subsection{Metadatos}

Sitic usa ficheros con encabezados para definir los metadatos, comúnmente llamados \texttt{front matter}.
Sitic usa la organización proporcionada para su contenido para minimizar cualquier
configuración adicional, aunque se puede sobrescribir en los metadatos.

Estos metadatos se identificarán a partir de unos caracteres delimitadores, los formatos soportados son:

\begin{itemize}
    \item \textbf{TOML}: identificado por +++.
    \item \textbf{YAML}: identificado por ---.
\end{itemize}

\textbf{Ejemplo de TOML}:

\begin{verbatim}
    +++
    title='Post 2'
    description='Post 2 description'
    tags=['tag1', 'tag2', 'tag3']
    +++

    El contenido va aquí
\end{verbatim}

\textbf{Ejemplo de YAML}:

\begin{verbatim}
    ---
    title: 'Post 2'
    description: 'Post 2 description'
    tags: ['tag1', 'tag2', 'tag3']
    ---

    El contenido va aquí
\end{verbatim}

% TODO
% \subsubsection{Variables}

\subsection{Secciones}

Sitic está hecho de forma que la organización su contenido tenga un propósito. La estructura usada
para organizar el contenido de origen se utiliza para organizar el sitio generado. Siguiendo
este patrón Sitic utiliza el nivel superior de su organización de contenido como la Sección.

Sitic creará automáticamente páginas para cada raíz de sección que enumere todo el contenido
de esa sección. Consulte la sección dedicada a las plantillas para obtener detalles sobre
la personalización de la forma en que aparecen.

Las páginas de sección también pueden tener un fichero de contenido con metadatos. Para ello basta con
definir un fichero con el nombre \texttt{index.md}, se tomará como el contenido propio de la sección en
lugar de un contenido de la sección.

\subsubsection{Secciones como ficheros}

Puede existir ocasiones en las que queramos tener una sección que carecerá de contenidos adicionales,
es decir, que no tendrá otros contenidos relacionados que se listarán dentro de la página de la sección.

En ese caso, cualquier fichero definido en la raíz del directorio de contenidos, se tomará como una sección
en la que se podrá definir directamente el contenido y metadatos sin necesidad de seguir el proceso de creación
de carpetas y fichero \texttt{index.md}.

\subsection{Taxonomías}

Sitic también ofrece la posibilidad del uso de taxonomías, por defecto existen dos tipos de taxonomías,
las \texttt{tags} y \texttt{categories}. Pero Sitic también ofrece al usuario la opción de configurar
nuevas taxonomías desde el fichero general de configuración.

\subsection{Definiendo nuevas taxonomías}

Para configurar nuevas taxonomías debemos hacerlo, añadiendo un nuevo elemento \texttt{taxonomies}
a la configuración principal, junto
con las definiciones en plural y singular de las nuevas taxonomías, un ejemplo, sería el siguiente:

\begin{verbatim}
    taxonomies:
        tag: "tags"
        category: "categories"
        series: "series"
\end{verbatim}

\subsection{Relacionando taxonomías y contenidos}

Una vez que las taxonomías ya están definidas en la configuración, la forma en la que se deben de relacionar
con los contenidos, es a través de los metadatos, creando una variable con el nombre plural definido de la
taxonomía, seguido de cada uno de los términos de la taxonomía a la que pertenece el contenido.

A continuación se muestra un ejemplo usando las taxonomías \texttt{tags} y \texttt{categories}, en un fichero
con los metadatos en formato \texttt{TOML}:

\begin{verbatim}
    +++
    title='Post 2'
    description='Post 2 description'
    tags=['tag1', 'tag2', 'tag3']
    categories=['cat1', 'cat3']
    +++

    El contenido va aquí
\end{verbatim}


\subsection{Página inicial}

La página inicial de una web, suele ser de un aspecto distinto al del resto de secciones o contenidos
de la web, para añadir un fichero con contenidos y metadatos, a la página inicial de un sitio
construido con Sitic, simplemente tendremos que añadir un fichero \texttt{index.md} en la raíz de
la carpeta de contenidos.

%%%%%%%%%%%%%%%%%%%%%%%%%%%%%%%%%%%%%%%%%%%%%%%%%%%%%%
%%%%%%%%%%%%%%%%%%%%%%%%%%%%%%%%%%%%%%%%%%%%%%%%%%%%%%

\section{Plantillas}

Sitic usa la librería \texttt{Jinja2}~\cite{jinja} como motor de plantillas. Tiene todas las funcionalidades
necesarias para construir plantillas de forma rápida y personalizable.

Si nunca has usado \texttt{Jinja2} anteriormente, te recomiendo visitar si página oficinal~\cite{jinja},
donde podrás ver todas las funcionalidades que ofrece y todas disponibles en Sitic.

Entre otras funcionalidades, ofrece la posibilidad de extender plantillas reduciendo así el número de
código duplicado entre plantillas.

\subsection{Tipos de plantillas}

Existen tres tipos de plantillas que se definen a continuación:

\begin{itemize}
    \item \textbf{Page}: Plantilla para renderizar un único contenido.
    \item \textbf{List}: Plantilla para listar contenidos, como pueden ser las secciones y las taxonomías.
    \item \textbf{Index}: La plantilla para la página inicial de tu web.
\end{itemize}

\subsection{Ruta de plantillas}

Como se ha comentado anteriormente, el directorio \texttt{templates}, es donde se deben de añadir todas
las plantillas. Por defecto un Sitio hecho con Sitic debe de tener un directorio \texttt{default}, con las
siguientes plantillas:

\begin{itemize}
    \item \textbf{page.html}: Plantilla por defecto para todas las páginas.
    \item \textbf{list.html}: Plantilla por defecto para secciones, taxonomía y homepage.
\end{itemize}

Pero Sitic te ofrece la posibilidad de personalizar las plantillas en función del tipo que sean.

\subsubsection{Page}

La plantilla para un contenido normal puede estar en cualquier de las siguientes rutas. Sitic buscará
la plantilla en el orden que se muestra a continuación:

\begin{itemize}
    \item /templates/\texttt{TYPE}/\texttt{TEMPLATE}.html
    \item /templates/\texttt{SECTION}/\texttt{TEMPLATE}.html
    \item /templates/\texttt{TYPE}/page.html
    \item /templates/\texttt{SECTION}/page.html
    \item /templates/default/page.html
\end{itemize}

Tanto \texttt{TYPE} como \texttt{TEMPLATE}, son los valores que se definan en los metadatos del contenido
en variable con el mismo nombre, totalmente en minúsculas.

En el caso de \texttt{SECTION}, se usa el nombre de la sección a la que pertenezca el contenido.

\subsubsection{Sección}

Para las secciones las rutas que Sitic busca son las siguientes:

\begin{itemize}
    \item /templates/section/\texttt{SECTION}.html
    \item /templates/\texttt{SECTION}/section.html
    \item /templates/\texttt{SECTION}/list.html
    \item /templates/default/section.html
    \item /templates/default/list.html
\end{itemize}

\subsubsection{Taxonomía}

En el caso de las taxonomias:

\begin{itemize}
    \item /templates/taxonomy/\texttt{SINGULAR}.html
    \item /templates/taxonomy/list.html
    \item /templates/default/taxonomy.html
    \item /templates/default/list.html
\end{itemize}

\subsubsection{Homepage}

Por último en la home:

\begin{itemize}
    \item /templates/index.html
    \item /templates/default/list.html
    \item /templates/default/page.html
\end{itemize}

\subsection{Elementos disponibles en las plantillas}

Cada plantilla dispondrá de una serie de elementos en función del tipo de plantilla de la que estemos
hablando.

Todas las plantillas tendrán acceso a la variable \texttt{site} en la que estarán todos los parámetros
presentes en la configuración del sitio y además tendrá acceso a la variable \texttt{node}, que representará
el nodo que se está renderizando, ya sea una página, sección o taxonomía. En este elementos están disponibles
todo los atributos de dicho elemento.

\subsection{Ejemplo}

A continuación se muestra una seria de ejemplo de uso de plantillas.

En primer lugar podemos ver la plantilla base de una web, con el nombre \textit{base.html}:

\begin{minted}{html}
    <!DOCTYPE html>
    <html lang="en">
      <head>
        <!-- Meta Tag -->
        <meta charset="UTF-8">
        <meta name="viewport" content="width=device-width, initial-scale=1.0">
        <meta http-equiv="X-UA-Compatible" content="IE=edge">

        <title>Personal Blog Template</title>

        <link rel="stylesheet" type="text/css" href="/static/css/style.css">
      </head>
      <body>

        <div id="main">
          <div class="container">
              <h2>
                
                    {{ node.title }}
                
                    My Blog
                
              </h2>
            </div>
            <div class="col-md-12 content-page">
              
              
            </div>
          </div>
        </div>

        <script type="text/javascript" src="/static/js/scripts.js"></script>
      </body>

    </html>
\end{minted}

Se puede apreciar en el documento anterior que es un documento HTML con ciertos peculiaridades del
sistema de plantillas \textit{Jinja2}.

En concreto podemos ver como se usan estructuras de control y como se define el bloque donde irá el contenido,
concretamente:

\begin{minted}{html}
    <div class="container">
        <h2>
          
              {{ node.title }}
          
              My Blog
          
        </h2>
      </div>
      <div class="col-md-12 content-page">
        
        
      </div>
    </div>
\end{minted}

A continuación podemos ver la plantilla de un contenido, extendiendo de la plantilla base y definiendo
el contenido necesario, así como haciendo uso de las taxonomías.

\begin{minted}{html}
    

    
    <div class="col-md-12 blog-post">

      <div class="post-title">
        <h1>{{ node.title }}</h1>
      </div>

      <div class="post-info">
        <span>
            {{ node.publication_date | datetimeformat("%d %B, %Y") }}
            
                / by <a href="#" target="_blank">{{ node.author }}</a>
            
        </span>
      </div>

      {{ node.content }}
    </div>
    
\end{minted}

Podemos ver que ya no es necesario volver a definir el \texttt{HTML} completo para cada uno de los
contenidos, sino que podemos extender de algunas de las plantillas ya existentes y definido qué
bloques vamos a completar.

En este caso se extiende de la plantilla \texttt{base.html} mostrada anteriormente y se define el 
contenido que tendrá el bloque denominado \texttt{content}, a su vez se accede a los distintos 
metadatos del contenido en este caso a los campos \texttt{title}, \texttt{author}, 
\texttt{publication\_date} y \texttt{content}, este último está compuesto del contenido de la página
en \texttt{HTML}.

%%%%%%%%%%%%%%%%%%%%%%%%%%%%%%%%%%%%%%%%%%%%%%%%%%%%%%
%%%%%%%%%%%%%%%%%%%%%%%%%%%%%%%%%%%%%%%%%%%%%%%%%%%%%%

\section{Internacionalización}

Sitic ofrece la posibilidad de tener una web multi-idioma, haciendo uso de la poderosa herramienta
gettext~\cite{gettext}, que Python soporta de base.

\subsection{Configuración}

Para que Sitic sepa que el sitio contendrá más de un idioma, es necesario configurarlo, a continuación
podemos ver un ejemplo:

\begin{verbatim}
    main_language: 'en'

    languages:
        en:
            name: 'English'
        es:
            name: 'Español'
\end{verbatim}

De esta forma configuramos dos idiomas, indicándole a Sitic que el idioma inglés, es el idioma principal
de la web. En el caso de que no se especificara qué idioma es el principal, Sitic tomaría cualquier de
los configurados como principal.

\subsection{Contenido}

A la hora de escribir contenidos en distinto idiomas, sólo es necesario añadir el idioma al que pertenece el
contenido en al nombre de éste, justo antes de la extensión del fichero. En el caso de que no se indique el
idioma, Sitic dará por hecho de que el contenido pertenece al idioma principal configurado.

En el siguiente ejemplo, se puede ver un ejemplo con una sección con contenidos en inglés y español:

\begin{verbatim}
    content/
    +-- section
    |   +-- content1.es.md
    +   +-- content1.md
\end{verbatim}

\subsection{Plantillas}

Para ver como traducir las cadenas presentes en las plantillas, los mejor es visitar la página oficial de
Jinja2~\cite{jinja}, donde se explican los distintos filtros disponibles.

\subsection{Generación y gestión}

Sitic ofrece la posibilidad de generar los ficheros .po y .mo, usado por gettext para realizar la traducción
de un sitio. Para ello ofrece dos comandos.

\paragraph{makemessages}

Este comando buscará todas las cadenas traducibles definidas en las plantillas y generará los ficheros .po
en la carpeta locales.

\begin{bashcode}
    $ sitic makemessages
\end{bashcode}

Un ejemplo de la carpeta locales generada por sitic:

\begin{verbatim}
    locales/
    +-- en
    |   +-- LC_MESSAGES
    |       +-- sitic.po
    +-- es
    |   +-- LC_MESSAGES
    +       +-- sitic.po
\end{verbatim}


\paragraph{compilemessages}

Una vez que hayamos completado los ficheros .po con las traducciones necesarias, es necesario compilar dichos
ficheros, para que gettext pueda obtener las traducciones correctamente. El comando para generar los ficheros
.mo necesario es:

\begin{bashcode}
    $ sitic compilemessages
\end{bashcode}

%%%%%%%%%%%%%%%%%%%%%%%%%%%%%%%%%%%%%%%%%%%%%%%%%%%%%%
%%%%%%%%%%%%%%%%%%%%%%%%%%%%%%%%%%%%%%%%%%%%%%%%%%%%%%

\section{Comandos}

En esta sección se indican los distintos comandos disponibles en sitic, aparte de los ya comentados
en secciones anteriores.

\paragraph{watch}

Con el objetivo de no tener que estar generando el sitio cada vez que se produce algún cambio en algunos
de los fichero que compone la web, está disponible el comando \texttt{watch}, que constantemente está
comprobando si existe algún cambio en algún fichero para regenerar el sitio.

\begin{bashcode}
    $ sitic watch
\end{bashcode}

\paragraph{servidor}

Con el siguiente comando, Sitic levanta un servidor local en el puerto que se desee, siendo por defecto
el 8000, de forma que el usuario pueda probar el funcionamiente de su web sin necesidad de instalar y
configurar ningún otro software:

\begin{bashcode}
    $ sitic server
\end{bashcode}

Este comando a su vez incluye la funcionalidad del comando \texttt{watch} anteriormente comentado. Este
comando está pensado para usarse durante el proceso de desarrollo de un sitio.

%%%%%%%%%%%%%%%%%%%%%%%%%%%%%%%%%%%%%%%%%%%%%%%%%%%%%%
%%%%%%%%%%%%%%%%%%%%%%%%%%%%%%%%%%%%%%%%%%%%%%%%%%%%%%

% TODO
% \section{Otras herramientas}
