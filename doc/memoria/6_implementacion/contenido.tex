Como complemento a la lectura de este capítulo se recomienda tener una copia
local del repositorio del proyecto, disponible para su libre descarga desde
el repositorio oficial.

\section{Entorno de construcción}

\subsection{Herramientas de diseño y desarrollo}

Como editor principal se ha utilizado Vim, una versión mejorada del editor de
texto vi, presente en todos los sistemas UNIX.

Se ha utilizado draw.io para la generación de diagramas de casos de uso,
secuencia, diagramas de flujo y similares. Se trata de una herramienta web, 
actualmente integrable en Google Drive, con un enorme número de elementos de
dibujo y la capacidad de exportar en un gran número de formatos, entre ellos en
formato vectorial PDF (Portable Document Format).

\subsection{Gestión de dependencias}

Para facilitar la gestión de las dependencias del proyecto se han utilizado VirtualEnv
y VirtualEnvWrapper. Estas herramientas permiten generar
entornos virtuales para cada proyecto, en los que se instalan las dependencias necesarias.
Estos entornos se activan y desactivan, de forma que las bibliotecas instaladas en
el entorno virtual de un proyecto no son accesibles desde el entorno del
otro proyecto. Esto evita la polución del nivel general de bibliotecas y facilita el
control estricto de las versiones de las dependencias.

Junto a virtualenv, la herramienta pip permite guardar en un fichero anexo
la lista de dependencias de un proyecto, de forma que sea fácil reinstalarlas todas
si hubiese que repetir la instalación en otro sistema.

\subsection{Control de versiones}

Todo el código fuente del proyecto se encuentra alojado en un repositorio público
en GitHub, haciendo uso de los planes gratuitos. GitHub es una repositorio que
utiliza el sistema de control de versiones Git. Además del alojamiento de
código, GitHub provee numerosas funcionalidades adicionales, tanto a nivel
social (permitiendo a las repositorios tener followers, por ejemplo) como a nivel funcional
(ofreciendo sistemas de tickets, estadísticas, etcétera).

El uso de un control de versiones es fundamental por varios motivos. En primer
lugar, sirve como sistema de copia de seguridad. En segundo lugar, permite deshacer
cambios en el código que no funcionen bien, siendo siempre posible volver
atrás. Por último, sirve como cuaderno de bitácora improvisado, ya que se guarda
el historial de commits que el desarrollador va enviando junto a los mensajes,
siendo posible ver en una línea temporal el progreso del trabajo.

\subsection{Lenguaje de programación}

Como se ha comentado en numerosas ocasiones en la presente memoria, el lenguaje
de programación elegido para el desarrollo del proyecto es Python, un
lenguaje interpretado de alto nivel desarrollado por Guido Van Rossum en 1991.
Python soporta múltiples paradigmas de programación, desde la orientación a objetos
hasta la programación funcional, pasando por el clásico estilo imperativo. Su
principal uso ha sido como lenguaje de scripting, pero también tiene su hueco en
contextos más amplios como lenguaje principal.

Su facilidad de aprendizaje, lo simple de su sintaxis (basada en la indentación
para marcar los bloques) y su extensibilidad (sobre todo gracias al uso de métodos
mágicos y metaprogramación) han hecho que el lenguaje sea muy popular y su uso
en los últimos años se haya expandido enormemente.

\subsection{Bibliotecas de terceros}

En el proyecto se han utilizado un gran número de bibliotecas auxiliares para
facilitar el desarrollo y ampliar la funcionalidad de forma sencilla.

\paragraph{Jinja2}
Es un lenguaje de plantillas moderno y diseñado para Python, modelado a partir de las plantillas de Django.
Es rápido, ampliamente utilizado y seguro con el entorno de ejecución.

Características:

\begin{itemize}
    \item Sistema de escapado de HTML para la prevención XSS.
    \item Herencia de la plantillas.
    \item Compilación opcional de plantillas por adelantado.
    \item Fácil de depurar.
    \item Sintaxis configurable.
\end{itemize}

\paragraph{Watchdog}

API para monitorizar eventos en el sistema de ficheros.

\begin{itemize}
    \item Multiplataforma.
    \item Herramientas para responder a cambios en directorios.
\end{itemize}

\paragraph{Click}

Es un paquete de Python para crear interfaces de línea de comandos de una manera
sencilla. Es el "Kit de creación de interfaz de
línea de comandos". Es altamente configurable.

Caraterísticas:

\begin{itemize}
    \item Anidamiento arbitrario de comandos.
    \item Generación de ayuda automática.
    \item Soporta la carga de subcomandos en tiempo de ejecución.
\end{itemize}

\paragraph{Six}

Proporciona utilidades simples para envolver las diferencias entre Python 2 y Python 3. Pretende
que soporte las bases de código que funcionan tanto en Python 2 como en 3 sin modificación.
Consta de un solo archivo Python, por lo que es muy sencillo de añadir a un proyecto.

%%%%%%%%%%%%%%%%%%%%%%%%%%%%%%%%%%%%%%%%%%%%%%%%%%%%%%
%%%%%%%%%%%%%%%%%%%%%%%%%%%%%%%%%%%%%%%%%%%%%%%%%%%%%%

\section{Organización del código fuente}

En esta sección se detalla la organización del código fuente del proyecto,
describiendo la utilidad de los ficheros y directorios.

El código fuente presente en la repositorio~\cite{repo} cuenta con los siguientes
elementos en el directorio raíz:

\begin{itemize}
    \item \texttt{doc}: directorio con toda la documentación del proyecto.
    \item \texttt{examples}: directorio con ejemplos de webs haciendo uso de la herramienta.
    \item \texttt{sitic}: código fuente.
    \item \texttt{README.md}: fichero de presentación para la repositorio.
    \item \texttt{requirements.txt}: fichero con dependencias instalables con Pip.
\end{itemize}

\subsection{Código fuente}

El código se encuentra íntegramente contenido en el directorio sitic del proyecto, que contiene lo siguiente:

\begin{itemize}
\item \texttt{cli parser.py}: definición de comandos disponibles.
\item \texttt{compilemessages.py}: comando para compilar traducciones.
\item \texttt{config.py}: encargado de parsea la configuración.
\item \texttt{content}: directorio con los ficheros que gestionan los contenidos.
    \begin{itemize}
        \item \texttt{base content.py}: implementa el contenido base del que heredan el resto de contenidos.
       \item \texttt{content factory.py}: implementa el patrón factory para obtener contenidos.
       \item \texttt{frontmatter handlers.py}: parseado de metadatos de contenidos.
       \item \texttt{homepage.py}: implementa la homepage.
       \item \texttt{menu builder.py}: implementa el menú.
       \item \texttt{menu.py}: implementa un elemento del menú.
       \item \texttt{page parser.py}: implementa el parseador de contenidos.
       \item \texttt{page.py}: implementa el contenido escrito por el usuario.
       \item \texttt{paginable content.py}: implementación de contenidos paginables.
       \item \texttt{paginator.py}: implementa paginación de secciones y taxonomías.
       \item \texttt{rss.py}: implementa el rss.
       \item \texttt{section.py}: implementa las secciones.
       \item \texttt{sitemap.py}: implementa el sitemap de la web.
       \item \texttt{taxonomy.py}: implementa las taxonomías.
    \end{itemize}

\item \texttt{files}: directorio con plantillas necesarios.
    \begin{itemize}
        \item \texttt{pot template.jinja}: plantilla de fichero de traducciones.
       \item \texttt{rss.xml}: plantilla para la generación del RSS.
       \item \texttt{search.js}: fichero javascript necesario para la búsqueda.
       \item \texttt{sitemap.xml}: plantilla de sitemap.xml.
    \end{itemize}

\item \texttt{generator.py}: generador de contenidos.
\item \texttt{logging.py}: logging de la aplicación.
\item \texttt{makemessages.py}: comando para obtener todos los mensajes traducibles del sitio web.
\item \texttt{scoper.py}: implementa funcionalidad para modificación de variables en las plantillas.
\item \texttt{search.py}: implementa el buscador.
\item \texttt{server.py}: comando para montar servidor local de pruebas.
\item \texttt{template}: directorio con el código encargado de la renderización de las plantillas.
    \begin{itemize}
        \item \texttt{filters.py}: filtros propios para usar en plantillas.
       \item \texttt{render.py}: renderizador de plantillas.
    \end{itemize}

\item \texttt{utils}: directorio con código común entre distintos módulos.
    \begin{itemize}
        \item \texttt{constants.py}: constantes del proyecto.
       \item \texttt{enums.py}: enumerados usados en el proyecto.
    \end{itemize}

\item \texttt{watcher.py}: Comando encargado de regenerar el contenido cada vez que existe algún cambio.
\end{itemize}

%%%%%%%%%%%%%%%%%%%%%%%%%%%%%%%%%%%%%%%%%%%%%%%%%%%%%%
%%%%%%%%%%%%%%%%%%%%%%%%%%%%%%%%%%%%%%%%%%%%%%%%%%%%%%

\section{Detalles de implementación}

En esta sección se describen algunos detalles en la implementación del proyecto
que han resultado de interés suficiente para ser dignos de mención, ya sea por los
conflictos que dieron, por su complejidad o por otras razones subjetivas.

\subsubsection{Implementación estructura de ficheros}

Uno de los primeros retos a la hora de la implementación, era ofrecer al usuario una forma sencilla de dividir
el contenido de los ficheros y los metadatos de estos.

Para que fuera sencillo tanto para los usuario, como para la herramienta poder diferencias esas dos partes
de forma inequívoca, se decidió que los ficheros de contenidos usará delimitadores que marcaran que
parte eran metadatos y debía ser tratados como tal y que parte era el contenido real del fichero.

Dado que el sistema ofrece escribir los metadatos en dos formatos, \texttt{YAML} y \texttt{TOML}, se decidió que cada
uno tuviera su propio carácter delimitador:

\begin{itemize}
    \item El formato \texttt{TOML} usaría el carácter +.
    \item El formato \texttt{YAML} usaría el carácter -.
\end{itemize}

A continuación se puede ver un ejemplo de un fichero de contenido \texttt{Markdown} con los metadatos en formato
\texttt{TOML}:

\begin{verbatim}
++++++++++++++++
title='Post 1'
tags=['tag1', 'tag2']
description="Post 1 description"
++++++++++++++++

## Variarum loricaeque

Lorem markdownum mihi. Avidoque doloris stabit; omen turaque est, ante ergo
nostraque, tibi.

    cpl_monitor_clean += html_window(management_ddl * 3, hacker);
    ccd -= -3;
    alignment_sdram += netbios_optical_party.webHardening(recycle_horizontal(5,
            networkCmsOpen, bandwidth_computer - text), -1);

## Curam vires adhuc domum

Tendere posuisse fatorum pharetras e diversis *lacertos*: ius cum vacuaque denos
quem, ausa transit puero. Undique foret et restatque supremaque accessi utile
mollire [bracchia Othryn](http://www.ratibus.net/curvavit.html) manumque; vimque
exprobravit sola ferrum diverso inania terramque. Qui illic et Romulus, ad erat
Pergama daedalus annis, fudit.

> Hunc medio, Argus plus pro lecto nimiumque dextera. Inde pondere ipse,
> lacertos [Troianae](http://faciunt.net/) habet rex, membra, lacrimisque
> onerosa. Dedi grata vani visurus corpora pariter ferarum **est quodque
> stimuletur**, vestri discrimina sceptrum, abit doceri septem iuvenumque. Ego
> est forte inpetus coniugis, auras inania regnat, tum languescuntque Byblida
> nisi, mentes. Vident candidus flectitur humana.
\end{verbatim}

Internamente el sistema automáticamente divide el contenido del fichero en dos partes y las procesa por separado.
A continuación se puede ver el fragmento de código que se encarga de realizar dicha tarea:

\begin{pythoncode}
# -*- condig: utf-8 -*-

from sitic.content.frontmatter_handlers import YamlHander, TomlHandler


handlers = [handler() for handler in [YamlHander, TomlHandler]]

def __get_handler(text):
    text_handler = None

    for handler in handlers:
        if handler.has_format(text):
            text_handler = handler
            break

    return handler


def __parse(text, handler):
    text = text.strip()
    metadata = {}

    handler = handler or __get_handler(text)
    if not handler:
        return metadata, text


    try:
        metadata, text = handler.split(text)
    except ValueError:
        return metadata, text

    metadata = handler.load(metadata)

    return metadata, text


def load(file_path, handler=None):
    text = ''

    with open(file_path, 'r') as f:
        text = f.read()

    text = text.strip()
    handler = handler or __get_handler(text)
    return loads(text, handler)
\end{pythoncode}

\subsubsection{Implementación de búsqueda}

Hoy en día, cualquier página web de un tamaño considerable con una cantidad de contenidos lo suficientemente
grande como para ser difícil encontrar cualquier contenido antiguo navegador por la misma, necesita un
buscador con el que se puedan encontrar de forma sencilla los contenidos.

Dado que \texttt{Sitic} genera un sitio web completamente estático, lo que implica que no tenemos ningún
servidor al que hacerle peticiones con los términos que se desean buscar, se decidió implementar
una búsqueda acorde a las características de la herramienta.

Para conseguirlo, la implementación consistía en varias partes, la primera de ellas, generar un índice de contenidos
en un fichero JSON que incluyera todos los contenidos publicados.

EL JSON resultante tendría el siguiente formato:

\begin{jsoncode}
[
    {
        "content": "Variarum loricaeque\nLorem markdownum mihi...",
        "title": "Post 1",
        "url": "/blog/post1"
    },
    {
        "content": "Variarum loricaeque\nLorem markdownum mihi...",
        "title": "Post 2",
        "url": "/blog/post2"
    },
    {
        "content": "Variarum loricaeque\nLorem markdownum mihi...",
        "title": "Post 6",
        "url": "/blog/post6"
    },
    {
        "content": "Variarum loricaeque\nLorem markdownum mihi...",
        "title": "Post 5",
        "url": "/blog/post5"
    },
    {
        "content": "Variarum loricaeque\nLorem markdownum mihi. Avct...",
        "title": "Post 3",
        "url": "/blog/post3"
    },
    {
        "content": "This is a Heading 1\n\n\tThis might be an intro...",
        "title": "Post 7",
        "url": "/blog/post7"
    },
    {
        "content": "\nBasic self-documenting ReStructured Text exa...",
        "title": "Post 8",
        "url": "/blog/post8"
    },
]
\end{jsoncode}

Este fichero se añadiría a la carpeta del sitio generado.

La tercera parte requería implementar la lógica necesaria en \texttt{Javascript}que usara el indice generado
y devolviera los resultados encontrados. Para ellos se hizo usa de la biblioteca para \texttt{Lunr.js} para
\texttt{Javascript} que permite realizar búsquedas en índices adaptados a su formato.

La tercera y última parte de la implementación consiste en que el usuario pueda de forma sencilla añadir dicho buscador en
su web, para ello, se debe de añadir un formulario con un id específico que el código Javascript anteriormente
comentado consultará y por último generar una plantilla llamada \texttt{search-results.js} con un bloque
determinado donde se mostrarán los resultados de la búsqueda.

\subsubsection{Implementación internacionalización}

Para ofrecer internacionalización en las posibles webs que se desarrollen usando \texttt{Sitic}, se ha hecho uso
de la famosa biblioteca \texttt{gettext}, que es compatible con el motor de templates usado \texttt{Jinja2}.

En primer lugar, el usuario debe de poder configurar qué idiomas van a estar disponibles en su web. Esto debe hacerlo
en el fichero configuración del sitio. A continuación podéis ver una configuración para dos idiomas, estableciendo como
idioma principal el inglés:

\begin{verbatim}
main_language: 'en'

languages:
    en:
		title: "English"
    es:
		title: "Español"
\end{verbatim}

Una vez hecho esto, el usuario deberá indicar al sistema de que idioma se trata cada contenido, añadiendo como sufijo
justo antes de la extensión, el código del idioma, en el caso de que no se añada ninguno, el sistema supondrá que
el formato del fichero es el idioma marcado como principal.

En el caso de las templates, el usuario tiene disponible un filtro propio de \texttt{Jinja2}. A continuación
se puede ver el uso di dicho filtro en una template:

\begin{minted}{html}
<!DOCTYPE html>
<html>
  <head>
    
    <title> - My Webpage</title>
    
  </head>
  <body>
    <div id="header">This is a test</div>
    <p>Hello {{ user }}!</p>
    <p>
      
      There is {{ count }} {{ name }} object.
      
      There are {{ count }} {{ name }} objects.
      
    </p>
    <p>{{ gettext('Hello World!') }}</p>
    <p>{{ ngettext('%(num)d apple', '%(num)d apples', apples|count) }}</p>
    <p>{{ _('Hello World!') }}</p>
    <p>this is multine
    translation</p>
    <p>
    This is is another multiline{{ test }}
    </p>
    <div id="content"></div>
    <div id="footer">
      
      &copy; Hey!! <a href="#">Click me!</a>.
      
    </div>
    
    
  </body>
</html>
\end{minted}

Por último, si el usuario tuviera que crear y manipular de forma totalmente manual los ficheros
necesarios por \texttt{gettext}, esto sería de poca utilidad. Por ello se implementaron dos comando
que hiciera la vida mucho más fácil al usuario:

\begin{itemize}
	\item \textbf{makemessages}: Extrae las cadenas de traducciones y genera los ficheros .po necesarios.
	\item \textbf{compilemessages}: Compila los ficheros .po en .mo usados por gettext.
\end{itemize}

A continuación se puede ver un fragmento del código de \texttt{makemessages}:

\begin{pythoncode}
def run(self):

    self.extract_translations()
    self.generate_pot_file()
    self.generate_po_file()
    self.delete_pot_file()

def extract_translations(self):
    for root, directory, files in os.walk(config.templates_path):
        for f in files:
            if not f.endswith(tuple(constants.VALID_TEMPLATES_EXTENSIONS)):
                continue
            file_path = os.path.join(root, f)
            relative_file_path = file_path.replace(config.base_path, '')
            with open(file_path, 'r') as content:
                translations = self.env.extract_translations(content.read())
                for t in translations:
                    t = list(t)
                    t.insert(0, relative_file_path)
                    self.translations.append(t)

def generate_pot_file(self):
    pot_template = self.env.get_template('pot_template.jinja')
    pot_content = pot_template.render({'translations': self.translations})

    if not os.path.isdir(config.locales_path):
        os.makedirs(config.locales_path)

    with open(self.pot_file_path, 'w', encoding='utf-8') as f:
        f.write(pot_content)

    msguniq_args = ['msguniq'] + self.msguniq_options + [self.pot_file_path]
    output, errors, status = call_subprocess(msguniq_args)

    if status != constants.STATUS_OK and errors:
        logger.error("errors happened while running msguniq\n{}".format(errors))
        sys.exit(1)
    elif errors:
        logger.error(errors)

    # Replace pot file with msguniq output
    with open(self.pot_file_path, 'w', encoding='utf-8') as f:
        f.write(output)
\end{pythoncode}
