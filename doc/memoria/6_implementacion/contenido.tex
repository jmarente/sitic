Como complemento a la lectura de este capítulo se recomienda tener una copia
local del repositorio del proyecto, disponible para su libre descarga desde
el repositorio oficial.

\section{Entorno de construcción}

\subsection{Herramientas de diseño y desarrollo}

\subsection{Gestión de dependencias}

Para facilitar la gestión de las dependencias del proyecto se han utilizado VirtualEnv
y VirtualEnvWrapper. Estas herramientas permiten generar
entornos virtuales para cada proyecto, en los que se instalan las dependencias necesarias.
Estos entornos se activan y desactivan, de forma que las bibliotecas instaladas en
el entorno virtual de un proyecto no son accesibles desde el entorno del
otro proyecto. Esto evita la polución del nivel general de bibliotecas y facilita el
control estricto de las versiones de las dependencias.

Junto a virtualenv, la herramienta pip permite guardar en un fichero anexo
la lista de dependencias de un proyecto, de forma que sea fácil reinstalarlas todas
si hubiese que repetir la instalación en otro sistema.

\subsection{Control de versiones}

Todo el código fuente del proyecto se encuentra alojado en un repositorio público
en GitHub, haciendo uso de los planes gratuitos. GitHub es una repositorio que
utiliza el sistema de control de versiones Git. Además del alojamiento de
código, GitHub provee numerosas funcionalidades adicionales, tanto a nivel
social (permitiendo a las repositorios tener followers, por ejemplo) como a nivel funcional
(ofreciendo sistemas de tickets, estadísticas, etcétera).

El uso de un control de versiones es fundamental por varios motivos. En primer
lugar, sirve como sistema de copia de seguridad. En segundo lugar, permite deshacer
cambios en el código que no funcionen bien, siendo siempre posible volver
atrás. Por último, sirve como cuaderno de bitácora improvisado, ya que se guarda
el historial de commits que el desarrollador va enviando junto a los mensajes,
siendo posible ver en una línea temporal el progreso del trabajo.

\subsection{Lenguaje de programación}

Como se ha comentado en numerosas ocasiones en la presente memoria, el lenguaje
de programación elegido para el desarrollo del proyecto es Python, un
lenguaje interpretado de alto nivel desarrollado por Guido Van Rossum en 1991.
Python soporta múltiples paradigmas de programación, desde la orientación a objetos
hasta la programación funcional, pasando por el clásico estilo imperativo. Su
principal uso ha sido como lenguaje de scripting, pero también tiene su hueco en
contextos más amplios como lenguaje principal.

Su facilidad de aprendizaje, lo simple de su sintaxis (basada en la indentación
para marcar los bloques) y su extensibilidad (sobre todo gracias al uso de métodos
mágicos y metaprogramación) han hecho que el lenguaje sea muy popular y su uso
en los últimos años se haya expandido enormemente.

\subsection{Bibliotecas de terceros}

En el proyecto se han utilizado un gran número de bibliotecas auxiliares para
facilitar el desarrollo y ampliar la funcionalidad de forma sencilla.

\paragraph{Jinja2}
Es un lenguaje de plantillas moderno y diseñado para Python, modelado a partir de las plantillas de Django.
Es rápido, ampliamente utilizado y seguro con el entorno de ejecución.

Características:

\begin{itemize}
    \item Sistema de escapado de HTML para la prevención XSS.
    \item Herencia de la plantillas.
    \item Compilación opcional de plantillas por adelantado.
    \item Fácil de depurar.
    \item Sintaxis configurable.
\end{itemize}

\paragraph{Watchdog}

API para monitorizar eventos en el sistema de ficheros.

\begin{itemize}
    \item Multiplataforma.
    \item Herramientas para responder a cambios en directorios.
\end{itemize}

\paragraph{Click}

Es un paquete de Python para crear interfaces de línea de comandos de una manera
sencilla. Es el "Kit de creación de interfaz de
línea de comandos". Es altamente configurable.

Caraterísticas:

\begin{itemize}
    \item Anidamiento arbitrario de comandos.
    \item Generación de ayuda automática.
    \item Soporta la carga de subcomandos en tiempo de ejecución.
\end{itemize}

\paragraph{Six}

Proporciona utilidades simples para envolver las diferencias entre Python 2 y Python 3. Pretende
que soporte las bases de código que funcionan tanto en Python 2 como en 3 sin modificación.
Consta de un solo archivo Python, por lo que es muy sencillo de añadir a un proyecto.

%%%%%%%%%%%%%%%%%%%%%%%%%%%%%%%%%%%%%%%%%%%%%%%%%%%%%%
%%%%%%%%%%%%%%%%%%%%%%%%%%%%%%%%%%%%%%%%%%%%%%%%%%%%%%

\section{Organización del código fuente}

En esta sección se detalla la organización del código fuente del proyecto,
describiendo la utilidad de los ficheros y directorios.

El código fuente presente en la repositorio~\cite{repo} cuenta con los siguientes
elementos en el directorio raíz:

\begin{itemize}
    \item \texttt{doc}: directorio con toda la documentación del proyecto.
    \item \texttt{examples}: directorio con ejemplos de webs haciendo uso de la herramienta.
    \item \texttt{sitic}: código fuente.
    \item \texttt{README.md}: fichero de presentación para la repositorio.
    \item \texttt{requirements.txt}: fichero con dependencias instalables con Pip.
\end{itemize}

\subsection{Código fuente}

El código se encuentra íntegramente contenido en el directorio sitic del proyecto, que contiene lo siguiente:

\begin{itemize}
\item \texttt{cli parser.py}: definición de comandos disponibles.
\item \texttt{compilemessages.py}: comando para compilar traducciones.
\item \texttt{config.py}: encargado de parsea la configuración.
\item \texttt{content}: directorio con los ficheros que gestionan los contenidos.
    \begin{itemize}
        \item \texttt{base content.py}: implementa el contenido base del que heredan el resto de contenidos.
       \item \texttt{content factory.py}: implementa el patrón factory para obtener contenidos.
       \item \texttt{frontmatter handlers.py}: parseado de metadatos de contenidos.
       \item \texttt{homepage.py}: implementa la homepage.
       \item \texttt{menu builder.py}: implementa el menú.
       \item \texttt{menu.py}: implementa un elemento del menú.
       \item \texttt{page parser.py}: implementa el parseador de contenidos.
       \item \texttt{page.py}: implementa el contenido escrito por el usuario.
       \item \texttt{paginable content.py}: implementación de contenidos paginables.
       \item \texttt{paginator.py}: implementa paginación de secciontes y taxonomías.
       \item \texttt{rss.py}: implementa el rss.
       \item \texttt{section.py}: implementa las secciones.
       \item \texttt{sitemap.py}: implementa el sitemap de la web.
       \item \texttt{taxonomy.py}: implementa las taxonomías.
    \end{itemize}

\item \texttt{files}: directorio con plantillas necesarios.
    \begin{itemize}
        \item \texttt{pot template.jinja}: plantilla de fichero de traducciones.
       \item \texttt{rss.xml}: plantilla para la generación del RSS.
       \item \texttt{search.js}: fichero javascript necesario para la búsqueda.
       \item \texttt{sitemap.xml}: plantilla de sitemap.xml.
    \end{itemize}

\item \texttt{generator.py}: generador de contenidos.
\item \texttt{logging.py}: logging de la aplicación.
\item \texttt{makemessages.py}: comando para obtener todos los mensajes traducibles del sitio web.
\item \texttt{scoper.py}: implementa funcionalidad para modificación de variables en las plantillas.
\item \texttt{search.py}: implementa el buscador.
\item \texttt{server.py}: comando para montar servidor local de pruebas.
\item \texttt{template}: directorio con el código encargado de la renderización de las plantillas.
    \begin{itemize}
        \item \texttt{filters.py}: filtros propios para usar en plantillas.
       \item \texttt{render.py}: renderizador de plantillas.
    \end{itemize}

\item \texttt{utils}: directorio con código común entre distintos módulos.
    \begin{itemize}
        \item \texttt{constants.py}: constantes del proyecto.
       \item \texttt{enums.py}: enumerados usados en el proyecto.
    \end{itemize}

\item \texttt{watcher.py}: Comando encargado de regenerar el contenido cada vez que existe algún cambio.
\end{itemize}

%%%%%%%%%%%%%%%%%%%%%%%%%%%%%%%%%%%%%%%%%%%%%%%%%%%%%%
%%%%%%%%%%%%%%%%%%%%%%%%%%%%%%%%%%%%%%%%%%%%%%%%%%%%%%

\section{Detalles de implementación}

\subsubsection{Implementación de búsqueda}

\subsubsection{Implementación de taxonomías}
