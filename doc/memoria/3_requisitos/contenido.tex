En este capítulo procederemos a describir de manera formal los objetivos que se
presentaron de manera genérica en el apartado 1.3, Objetivos.

En particular, se detallan:

\begin{itemize}
\item Los objetivos del sistema, que resumen a muy grandes rasgos la funcionalidad del proyecto.
\item Los requisitos funcionales, que definen las funciones del sistema software
y sus componentes.
\item Los requisitos de información, que detallan los datos que el sistema usará
durante el desarrollo y ejecución para llevar a cabo de forma adecuada su
funcionalidad.
\item Los requisitos no funcionales, que, en general, especifican criterios de diversas
categorías que el software debe cumplir para garantizar un buen nivel
de calidad más allá de su funcionamiento.
\end{itemize}

Para terminar el capítulo se presentan diversas alternativas tecnológicas para cubrir
las necesidades surgidas en los requerimientos del sistema, detallando las
decisiones tomadas.

%%%%%%%%%%%%%%%%%%%%%%%%%%%%%%%%%%%%%%%%%%%%%%%%%%%%%%
%%%%%%%%%%%%%%%%%%%%%%%%%%%%%%%%%%%%%%%%%%%%%%%%%%%%%%

\section{Objetivos del sistema}

El objetivo principal de Sitic, definido a alto nivel es el siguientes:

\begin{center}

    \begin{tabularx}{\textwidth}{|c|X|}
        \hline
        Título & Crear contenido HTML a partir ficheros de texto plano\\

        \hline

        Descripción & Sitic permitirá generar una web completamente estática a partir
        de ficheros de texto plano, que se escribirán siguiendo alguno de los lenguajes de
        marcado soportados. \\

        \hline
    \end{tabularx}
    \captionof{table}{OBJ-1, objetivo del sistema 1}
\end{center}

%%%%%%%%%%%%%%%%%%%%%%%%%%%%%%%%%%%%%%%%%%%%%%%%%%%%%%
%%%%%%%%%%%%%%%%%%%%%%%%%%%%%%%%%%%%%%%%%%%%%%%%%%%%%%

\section{Catálogo de requisitos}

Se presentan a continuación los requisitos del sistema, tanto a nivel funcional
como de información y no funcionales.

\subsection{Requisitos funcionales}

En esta subsección se introducen los requisitos funcionales, que describen las funciones
del sistema software.

\begin{center}

    \begin{tabularx}{\textwidth}{|c|X|}
        \hline
        Título & Generación de HTML a partie de ficheros\\

        \hline

        Descripción & El sistema debe de poder generar páginas HTMLs a partir
        del contenido de ficheros de texto plano.\\

        \hline
    \end{tabularx}
    \captionof{table}{RQF-1, requisito funcional 1: generación}
\end{center}

\begin{center}

    \begin{tabularx}{\textwidth}{|c|X|}
        \hline
        Título & Internacionalización\\

        \hline

        Descripción & Se deberán de poder configurar tantos idiomas con se desee
        generando sitios independientes por idiomas de sólo aquellos contenidos
        en el idioma generado.\\

        \hline
    \end{tabularx}
    \captionof{table}{RQF-2, requisito funcional 2: internacionalización}
\end{center}

\begin{center}

    \begin{tabularx}{\textwidth}{|c|X|}
        \hline
        Título & Búsqueda\\

        \hline

        Descripción & Debe de ser posible realizar búsquedas de contenidos en la web generada.\\

        \hline
    \end{tabularx}
    \captionof{table}{RQF-3, requisito funcional 3: búsqueda}
\end{center}

\begin{center}

    \begin{tabularx}{\textwidth}{|c|X|}
        \hline
        Título & Categorización\\

        \hline

        Descripción & Los contenidos deben de poder ser categorizable en secciones y taxonomías,
        dando como resultado grupos de contenidos en el sitio generado.\\

        \hline
    \end{tabularx}
    \captionof{table}{RQF-4, requisito funcional 4: categorización}
\end{center}

\begin{center}

    \begin{tabularx}{\textwidth}{|c|X|}
        \hline
        Título & Paginación\\

        \hline

        Descripción & La herramienta debe de proporcional paginación en los contenido agrupados, 
        evitando así generar páginas demasiado grandes.\\

        \hline
    \end{tabularx}
    \captionof{table}{RQF-5, requisito funcional 5: paginación}
\end{center}

\subsection{Requisitos de información}

A continuación se presentan los requisitos de información, que describen los datos
necesarios que requiere la aplicación para operar.

\begin{center}

    \begin{tabularx}{\textwidth}{|c|X|}
        \hline
        Título & Configuración del sitio\\

        \hline

        Descripción & El sitio tendrá un fichero de configuración básica: 
        \begin{itemize}
            \item Idiomas
            \item Atributos referentes al sitio(paginación, urls, etc)
            \item Menús
        \end{itemize}\\

        \hline
    \end{tabularx}
    \captionof{table}{RQF-6, req. de información 6: configuración}
\end{center}

\begin{center}

    \begin{tabularx}{\textwidth}{|c|X|}
        \hline
        Título & Fichero con contenido\\

        \hline

        Descripción & Los ficheros de contenido se compondrá de:
        \begin{itemize}
            \item Metadatos
            \item Contenido
            \item Menús
        \end{itemize}\\

        \hline
    \end{tabularx}
    \captionof{table}{RQF-7, req. de información 7: fichero}
\end{center}

\begin{center}

    \begin{tabularx}{\textwidth}{|c|X|}
        \hline
        Título & Plantillas HTML\\

        \hline

        Descripción & Plantillas que se usarán para renderizar los contenidos\\

        \hline
    \end{tabularx}
    \captionof{table}{RQF-8, req. de información 8: plantillas}
\end{center}

\subsection{Requisitos no funcionales}

Seguidamente se presentan los requisitos no funcionales, que especifican criterios
usados para juzgar la operación del software a través de varias categorías, más
allá de su comportamiento específico.

\subsubsection{Requisitos de fiabilidad}

\begin{center}

    \begin{tabularx}{\textwidth}{|c|X|}
        \hline
        Título & Generación robusta\\

        \hline

        Descripción & El sitio generado debería deberá ser completamente fiable
        en el sentido que se generen tantos contenidos como los definidos.\\

        \hline
    \end{tabularx}
    \captionof{table}{RQF-9, req, no funcional 9: generación robusta}
\end{center}

\subsubsection{Requisitos de accesibilidad y usabilidad}

\begin{center}

    \begin{tabularx}{\textwidth}{|c|X|}
        \hline
        Título & Alta capacidad de configuración\\

        \hline

        Descripción & El sistema deberá de ser tan configurable como sea
        posible, de forma que cualquier usuario lo pueda adaptar a sus
        necesidades lo máximo posible.\\

        \hline
    \end{tabularx}
    \captionof{table}{RQF-10, req. no funcional 10: configuración}
\end{center}

\subsubsection{Requisitos de eficiencia}

\begin{center}

    \begin{tabularx}{\textwidth}{|c|X|}
        \hline
        Título & Generación de gran cantidad de documentos\\

        \hline

        Descripción & El sistema debería de ser capaz de generar un sitio a
        partir de una gran cantidad de contenidos en un tiempo razonable.\\

        \hline
    \end{tabularx}
    \captionof{table}{RQF-11, req. no funcional 11: eficiencia }
\end{center}

%%%%%%%%%%%%%%%%%%%%%%%%%%%%%%%%%%%%%%%%%%%%%%%%%%%%%%
%%%%%%%%%%%%%%%%%%%%%%%%%%%%%%%%%%%%%%%%%%%%%%%%%%%%%%

\section{Alternativas de solución}

En esta sección, se presentan diferentes alternativas tecnológicas que permiten
satisfacer los requerimientos del sistema, presentando las alternativas elegidas y
detallando las razones de esta decisión.

\subsection{Biblioteca para crear simples interfaces por línea de comandos}

Dada la mayor familiaridad del desarrollador con el lenguaje, solo se barajaron
frameworks de desarrollo basados en el lenguaje Python.

\begin{itemize}
    \item \textbf{Click}~\cite{click}: paquete de Python para crear interfaces de línea de comandos hermosas de
una manera composible con el código tan pequeño como sea necesario. Es el "Kit de creación de interfaz de
línea de comandos". Es altamente configurable pero viene con los defectos sensatos fuera de la caja. Su objetivo es
hacer que el proceso de escritura de herramientas de línea de comandos sea rápido y divertido, al tiempo
que evita cualquier frustración causada por la imposibilidad de implementar una CLI API.
\item \textbf{Fire}~\cite{python-fire}: es una biblioteca para generar automáticamente interfaces de línea de comandos
desde cualquier objeto Python.
\item \textbf{Argparse}~\cite{argparse}: pertenece a la biblioteca estándar de python y facilita la escritura de interfaces
de línea de comandos fáciles de usar. El programa define qué argumentos requiere, y argparse descubrirá cómo
analizar los que están fuera de sys.argv. El módulo argparse también genera automáticamente mensajes de ayuda
y uso y emite errores cuando los usuarios dan al programa argumentos inválidos.
\end{itemize}

Se tomó la decisión de optar por \textbf{Click} ya que parecía más sencillo y las opciones que ofrecía sería mucho
más útiles.

\subsection{Sistema de templates}

El sistema contará con un motor de templates que permitirá al usuario definir y extenderlas de forma que pueda reutilizar
prácticamente todo el código HTML que genere para las propias templates de su sitio. Las alternativas barajadas son:


\begin{itemize}
\item \textbf{Jinja}~\cite{jinja}: es un motor de plantillas completo para Python. Cuenta con soporte unicode completo,
un entorno de ejecución de espacio de trabajo integrado opcional, ampliamente utilizado y licenciado por BSD.
\item \textbf{Chamaleon}~\cite{chamaleon}: es un motor de plantilla HTML / XML para Python. Está diseñado para generar
la salida de documentos de una aplicación web, normalmente marcado HTML o XML.
\end{itemize}

Finalmente se decidió usar \textbf{Jinja} por ser el sistema de template más robusto y con mayor número de utilidades.

\subsection{Lenguajes de marcado soportados}

El sistema convertirá los contenidos escritos en ficheros de texto plano a HTML, estos ficheros deberá de estar escritos
usando algún lenguaje de marcado. Las opciones que se barajaron fueron las siguientes:

\begin{itemize}
\item \textbf{Markdown}~\cite{markdown}: es un lenguaje de marcado ligero que trata de conseguir la máxima legibilidad y facilidad
de publicación tanto en su forma de entrada como de salida, inspirándose en muchas convenciones existentes para marcar mensajes
de correo electrónico usando texto plano. Se distribuye bajo licencia BSD.
\item \textbf{RestructuredText}~\cite{restructuredtext}: es un lenguaje de marcas ligero creado para escribir textos con formato definido de manera cómoda
y rápida. Es parte del proyecto Docutils dentro de la comunidad de Python. Es una evolución de Structured Text.
\item \textbf{Textile}~\cite{textile}: es un lenguaje de marcado ligero que partiendo de un texto con marcas genera código HTML.
Se utiliza para generar cualquier tipo de contenido escrito que deba ser publicado en internet.
\item \textbf{Wikitexto}~\cite{wikitexto}: es un texto elaborado mediante un lenguaje de marcado especial para la creación de páginas wikis,
en servidores que tengan instalado algún software para wikis. El lenguaje de marcado para wikis se denomina lenguaje wikitexto
y no existe un estándar que defina su sintaxis, sus características y su estructura, como la tiene el lenguaje HTML. Por el
contrario, depende del software wiki utilizado.
\end{itemize}

En un primer momento se decidió usar \textbf{Markdown}, ya que parecía el mas intuitivo, sencillo y el más usado por la comunidad. Finalmente,
se decidió también por dar soporte a \textbf{RestructuredText} y \textbf{Textile}, ofreciendo al usuario la posibilidad de elegir aquel
con el que se sienta más cómodo.

\subsection{Formato de configuración para los metadatos de los ficheros}

En cada fichero de texto plano, se podrá definir una serie de metadatos, usando un formato concreto. Las alternativas pensadas fueron
las siguientes:

\begin{itemize}
\item \textbf{JSON}~\cite{json}: es un formato de texto ligero para el intercambio de datos. JSON es un subconjunto de la notación literal de
objetos de JavaScript aunque hoy, debido a su amplia adopción como alternativa a XML, se considera un formato de lenguaje independiente.
\item \textbf{TOML}~\cite{toml}: es un formato de archivo de configuración que es fácil de leer debido a la semántica obvia que pretende ser
"mínima". TOML está diseñado para asignar sin ambigüedad a un diccionario.
\item \textbf{YAML}~\cite{yaml}: formato de serialización de datos legible por humanos inspirado en lenguajes como XML, C, Python, Perl,
\end{itemize}


Finalmente se decidió usar dos alternativas, tanto \textbf{YAML} como \textbf{TOML}, de forma que el usuario use aquel que encuentre más
sencillo.
