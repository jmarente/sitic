
Durante el transcurso del desarrollo de proyecto y sobre todo al término del mismo,
se han obtenido unas conclusiones y unos resultados, tanto de forma personal
como para con la comunidad, que intentaremos reflejar en este capítulo.

%%%%%%%%%%%%%%%%%%%%%%%%%%%%%%%%%%%%%%%%%%%%%%%%%%%%%%
%%%%%%%%%%%%%%%%%%%%%%%%%%%%%%%%%%%%%%%%%%%%%%%%%%%%%%

\section{Resumen de objetivos}

Es evidente que su realización no me ha dejado indiferente. No ha sido fácil construir una idea
clara sobre lo que se quería hacer. Así como solucionar los distintos problemas que han ido apareciendo
a lo largo del desarrollo.

También decir que el proyecto me ha ocupado bastante más tiempo del esperado en un principio. Tuve
muchos problemas y alguna que otra duda en algunas fases de desarrollo, que me tuvieron
bloqueado durante un tiempo hasta encontrar la solución más adecuada para estos. A pesar de todo,
estoy muy satisfecho con el resultado final.

Se puede decir que el proyecto goza de buena calidad. Se ha intentado hacer un software sencillo,
intuitivo, fácil de usar y entretenido para el usuario.

\section{Conclusiones personales}

Durante el desarrollo del proyecto se han aprendido muchísimas cosas: como hacer distintas ramas de
desarrollo, plantear y crear calendarios, usar las herramientas adecuadas, tomar decisiones importantes
durante el desarrollo, documentación del código, organización, etc. Ya que durante la carrera se
han realizado distintas prácticas y trabajos de complejidad, pero nada con el tamaño y duración que
requiere un Proyecto de fin de carrera. Una vez finalizado, creo que tengo la experiencia necesaria
para afrontar otro proyecto con buenos resultados.

Puedo decir que he profundizado y consolidado bastante en el lenguaje de programación Python.
Además he obtenido más práctica a la hora de manejar bibliotecas
externas, así como entender su documentación e integrarlas en un proyecto.

En definitiva, este proyecto me ha hecho madurar como persona, estudiante y profesional. He aprendido a buscar
bibliografía, opiniones en otras personas, compartir ideas, seguir un horario, cumplir una fechas de
entrega y enfrentarme a un proyecto de estas características.

\section{Mejoras y ampliaciones}

Las posibles mejoras y ampliaciones que se podrían añadir al proyecto en futuras versiones, se comentan
a continuación:

\begin{itemize}
    \item \textbf{Soporte Filtros/funciones propias} de forma que cada usuario puede personalizar aún mas
    funciones a usar en las plantillas.
    \item \textbf{Método watch que recargue automáticamente el navegador}: permitir que el modo watch recargue
    automáticamente el navegador cuando se detecte algún cambio.
    \item \textbf{Permitir temas}: permitir la creación de temas, es decir, poder usar temas definidos
    por otros usuarios y únicamente preocuparte por la redacción del contenido.
    \item \textbf{Permitir fichero con contenido para taxonomía}: permitir definir metadatos en las taxonomías
    tal y como se puede hacer en los contenidos y secciones.
    \item \textbf{Permitir publicar un solo idioma}: no verse obligado a generar el sitio completo y podre generar
    un único idioma o página.
\end{itemize}
