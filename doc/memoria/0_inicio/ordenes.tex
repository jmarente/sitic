% http://www.tex.ac.uk/cgi-bin/texfaq2html?label=chngmargonfly

\newenvironment{changemargin}[2]{%
  \begin{list}{}{%
    \setlength{\topsep}{0pt}%
    \setlength{\leftmargin}{#1}%
    \setlength{\rightmargin}{#2}%
    \setlength{\listparindent}{\parindent}%
    \setlength{\itemindent}{\parindent}%
    \setlength{\parsep}{\parskip}%
  }%
  \item[]}{\end{list}}

\newenvironment{nota}{
  \begin{changemargin}{2em}{2em}
    \textbf{\textsc{Nota: }}
}{
  \end{changemargin}
}

\newcommand{\nombreProyecto}{Sitic: framework para generar páginas web estáticas}
\newcommand{\jugador}{\textit{Jugador}}
\newcommand{\sistema}{\textit{Sistema}}


% %% PRODUCTOS
% \newcommand{\nombrepostprocesador}{ACL2:\colonhyp{}Procesador}
% \newcommand{\nombrevisor}{XMLEye}
% \newcommand{\nombreyaxml}{YAXML:\colonhyp{}Reverse}
% \newcommand{\postprocesador}{\texttt{\nombrepostprocesador}\xspace}
% \newcommand{\visor}{\nombrevisor\xspace}
% \newcommand{\yaxml}{\texttt{\nombreyaxml}\xspace}

% \newcommand{\biblioteca}[1]{\index{#1}\texttt{#1}}

% % YAML/YAXML
% \newcommand{\etiqueta}[1]{{\bfseries \ttfamily #1}}

% % ACL2

% \newcommand{\orden}[1]{\texttt{#1}}   % Nombre de orden ACL2
% \newcommand{\fichero}[1]{\texttt{#1}} % Nombre de fichero
% \newcommand{\evento}[1]{\texttt{#1}}  % Nombre de un evento
% \newcommand{\libro}[1]{\textsc{#1}}   % Libro ACL2

% % PERL

% \newcommand{\modulo}[1]{\index{módulo Perl!#1}\index{#1|see{módulo Perl!#1}}\texttt{#1}}
% \newcommand{\funcion}[1]{\textit{#1}}

% % JAVA

% \newcommand{\clase}[1]{\textit{#1}}   % Clase Java
% \newcommand{\metodo}[1]{\texttt{#1}}  % Método (también Perl)
% \newcommand{\paquete}[1]{\texttt{#1}}

% %% MANUALES

% \newcommand{\accesoteclado}[1]{\textsc{#1}} % Acceso de teclado

% %% OTROS

% \newcommand{\patron}[1]{\emph{#1}}

% \newcolumntype{,}{>{$}r<{$}}
% \newcommand{\Index}[1]{#1\emph{\index{#1}}}

% \newcounter{pasoacept}

% \newenvironment{pruebaaceptacion}{
%   \setcounter{pasoacept}{0}
%   \begin{center}
%   \begin{tabular}{| >{\stepcounter{pasoacept}\arabic{pasoacept}. }p{.4\linewidth} | p{.5\linewidth}|}
%     \hline
%     \multicolumn{1}{| c |}{\textbf{Paso seguido}} & \multicolumn{1}{c|}{\textbf{Resultado esperado}} \\
%     \hline
%     \hline
% }{
%   \hline
%   \end{tabular}
%   \end{center}
% }

% \renewcommand{\lstlistlistingname}{Listados}
% \renewcommand{\lstlistingname}{Listado}

% \newcommand{\CPP}
% {\mbox{C\hspace{-.1em}\raise.2ex\hbox{+\hspace{-.1em}+}}\xspace}

% \setlength{\extrarowheight}{4pt}

% \lstset{
%   extendedchars,
%   flexiblecolumns,
%   stringstyle=\ttfamily,
%   showstringspaces=false,
%   frame=tb
% }

% %% PARTE DE DOCBOOK

% \newcommand{\application}[1]{\index{#1}\emph{#1}}
% \newcommand{\cmdsynopsis}[1]{\nohyphens{\texttt{#1}}}
% \newcommand*{\command}[1]{\nohyphens{\textbf{\texttt{#1}}}}
% \newcommand{\constant}[1]{\texttt{#1}}
% \newcommand{\computeroutput}[1]{#1}
% \newcommand*{\email}[1]{\nohyphens{\texttt{#1}}}
% \newcommand*{\envar}[1]{\nohyphens{\texttt{#1}}}
% \newcommand*{\filename}[1]{\texttt{#1}}
% \newcommand*{\guibutton}[2][]{\emph{#2}}
% \newcommand*{\guilabel}[2][]{\emph{#2}}
% \newcommand*{\guimenuitem}[2][]{\emph{#2}}
% \newcommand*{\guimenu}[2][]{\emph{#2}}
% \newcommand*{\keycombo}[1]{\textsc{#1}}
% \newcommand*{\keysym}[1]{\textsc{#1}}
% \newcommand*{\option}[1]{\nohyphens{\texttt{#1}}}
% \newcommand{\prompt}[1]{#1}
% \newcommand{\varname}[1]{\nohyphens{\texttt{#1}}}

% \newcommand{\note}[1]{\vskip 1em
%     \fbox{\parbox{\textwidth}{\textsc{Nota}
%     \vskip 1em #1}} \vskip 1em}


%%% Local Variables:
%%% mode: latex
%%% TeX-master: "memoria"
%%% End:
