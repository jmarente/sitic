\section{Motivación}

La idea de desarrollar este proyecto surge a raíz de usar otras herramientas con un objetivo
parecido, pero que no llegaba a satisfacer todas las funcionalidades que necesitaba, viéndote obligado
a suplirlas con otras herramientas que no eran totalmente compatibles con las originales.

En noviembre de 2015 forme parte en el desarrollo de una página web totalmente estática con información
bastante importante a la que accederían personas de distintos idiomas. Para el desarrollo de la web mencionada,
se usó Hugo~\cite{hugo}, uno de los generadores estáticos más conocidos. Pero a pesar de ser uno de los más usados,
la herramienta carecía de funcionalidades que considerábamos totalmente imprescindibles en el desarrollo
de una plataforma tan importante.

De ahí nace la motivación para intentar desarrollar una herramienta que aunque siga los mismos principios,
añada funcionalidades de las que las herramientas actuales carecen y no tiene pensado añadir en un futuro
próximo.

También he de añadir que tras conocer abiertamente el mundo del Software libre, gracias a la importancia
que se le presta en la Universidad de Cádiz. Se decidió que el proyecto fuera software libre bajo
licencia GPL. Y así cualquier persona interesada en el proyecto y en el software libre
en general, pudiera usar los recursos del proyecto o colaborar libremente.

\section{Objetivos}

A la hora de definir los objetivos de un sistema, podemos agruparlos en dos tipos
diferentes: funcionales y transversales. Los primeros se refieren a qué debe hacer
la herramienta que vamos a desarrollar, e inciden directamente en la experiencia del
usuario y de potenciales desarrolladores.

Por otro lado, los objetivos transversales son aquellos invisibles al usuario final,
pero que de forma inherente actúan sobre el resultado final de la aplicación y
sobre la experiencia de desarrollo de la misma.

\subsection{Funcionales}
\begin{itemize}
\item Crear una herramienta que permita a los usuarios crear una web completamente estática
con todas las funcionalidades que una web normal y dinámica tienen.
\item Dar la posibilidad a los usuarios de personalizar en la medida de todo lo posible la configuración
inicial de la herramienta de forma que puedan adaptar lo máximo posible a sus necesidades.
\item Dar la posibilidad que cualquier usuario medio Linux pueda crear una web desde cero sin tener
ningún conocimiento de programación o de servidores web.
\end{itemize}

\subsection{Transversales}
\begin{itemize}
\item Aplicar mis conocimientos sobre el desarrollo web en general.
\item Adquirir soltura en el uso del lenguaje de programación Python en herramientas de terminal.
\item Utilizar un enfoque de análisis, diseño y codificación orientado a objetos,
de una forma lo más clara y modular posible, para permitir ampliaciones y
modificaciones sobre la aplicación por terceras personas.
\item Hacer uso de herramientas básicas en el desarrollo de software, como son los
sistemas de control de versiones para llevar un control realista del desarrollo
del software, así como hacer de las veces de sistema de copias de seguridad.
\end{itemize}

\section{Estructura del documento}

Este documento está compuesto por las siguientes partes:

\begin{itemize}
\item \textbf{Introducción}: pequeña descripción del proyecto, así como los objetivos y estructura del documento.

\item \textbf{Descripción general}: descripción más amplia sobre el proyecto, así como todas las características relevantes
que tendrá.

\item \textbf{Planificación}: exposición de la planificación del proyecto y las distintas etapas que esta compuesto el mismo.

\item \textbf{Análisis}: fase de análisis del sistema, empleando la metodología seleccionada. Se definirán los
requisitos funcionales del sistema, diagramas de caso de uso, diagramas de secuencia y contrato de las operaciones.

\item \textbf{Diseño}: realización del diseño del sistema, diagramas de secuencia y clases aplicadas al diseño.

\item \textbf{Implementación}: aspectos más relevantes durante la implementación del proyecto. Y problemas que han aparecido
durante el desarrollo.

\item \textbf{Pruebas}: pruebas realizadas a la aplicación, con el fin de comprobar su correcto funcionamiento y
cumplimiento de las expectativas.

\item \textbf{Conclusiones}: conclusiones obtenidas tras el desarrollo de la aplicación.

\item \textbf{Apéndices}:
\begin{itemize}
\item \textbf{Manual de instalación}: manual para la correcta instalación del proyecto en el sistema.
\item \textbf{Manual de usuario}: manual de usuario para el correcto uso de la aplicación.
\end{itemize}

\item \textbf{Licencia GPL}: texto completo sobre la licencia GPL, por la cual se rige el proyecto.

\item \textbf{Bibliografía}: libros y referencias usadas durante el desarrollo del proyecto.

\end{itemize}
