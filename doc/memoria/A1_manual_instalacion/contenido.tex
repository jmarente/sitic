A continuación se presentan las instrucciones de instalación de la herramienta.

Antes de continuar se presentan los requisitos de hardware y software mínimos
para la correcta ejecución de la plataforma web.

\paragraph{Hardware}

\begin{itemize}
\item 512MiB de memoria RAM como mínimo.
\item 10GiB de disco duro como mínimo.
\item Acceso a Internet.
\end{itemize}

\paragraph{Software}

\begin{itemize}
\item Sistema operativo \textbf{GNU/Linux}, preferiblemente basado en paquetería Debian.
\item Intérprete de \textbf{Python}, versión mínima 2.7 o 3.5.
\item Soporte de entornos virtuales \textbf{VirtualEnv} para la encapsulación de dependencias.
\end{itemize}

\section{Instalación}

\subsection{Descarga de código}

Para instalar la plataforma web es necesario clonar el repositorio~\cite{repositorio}
con el código de la aplicación. Primero, creamos una ubicación donde alojar el
código desde la terminal:

\begin{bashcode}
    $ mkdir ~/sitic -p
    $ cd ~/sitic
\end{bashcode}

Una vez ahí, instalamos el software de control de versiones \textbf{git} y
clonamos el repositorio desde Github:

\begin{bashcode}
    $ sudo apt-get install git
    $ git clone https://github.com/josemarente/sitic.git .--
\end{bashcode}

\subsubsection{Instalación de dependencias}

Como se ha comentado previamente, el proyecto utiliza \textbf{Virtualenv} y
\textbf{Virtualenvwrapper} para una gestión más limpia de las dependencias. La
instalación de estos dos elementos es sencilla, en primer lugar se instala el
gestor de paquetes y los dos paquetes mencionados:

\begin{bashcode}
    $ sudo apt-get install python-pip
    $ sudo pip install virtualenv
    $ sudo pip install virtualenvwrapper
\end{bashcode}

En segundo lugar, añadimos el código de \textit{bootstrapping} de
Virtualenvwrapper a nuestro perfil de terminal, habitualmente \texttt{.bashrc}:

\begin{bashcode}
    $ cat >> ~/.bashrc

    if [ -f /usr/local/bin/virtualenvwrapper.sh ]
    then
    source /usr/local/bin/virtualenvwrapper.sh
    fi

    EOF
\end{bashcode}

Hecho esto, será necesario reiniciar la terminal o volver a cargar el perfil del a terminal,
para el caso de \texttt{.bashrc}, se puede hacer de la siguiente forma:

\begin{bashcode}
    $ source ~/.bashrc
\end{bashcode}

Tras esto, podremos crear el entorno virtual e instalar las dependencias:

\begin{bashcode}
    $ mkvirtualenv sitic
    $ sudo apt-get install python-dev
    $ pip install -r requirements.txt
\end{bashcode}

El proceso de instalación de dependencias puede tardar entre 5 y 10 minutos aproximadamente.

\subsection{Comando sitic}

Por último sólo quedaría que sitic estuviera disponible en el sistema de forma global, es decir,
que podamos ejecutar su comando sin necesidad de en que ruta está y sin necesidad de poner la ruta completa
para lanzarlos, para ello bastaría con hacer un enlace simbólico del ejecutable de sitic en
\texttt{/usr/local/bin} de la siguiente manera:

\begin{bashcode}
    $ ln -s ~/sitic/main.py /usr/local/bin/sitic
\end{bashcode}

\section{Pruebas de implantación}

Para probar que la instalación es correcta, lo ideal es que pruebe la aplicación
a fondo, revisando toda la funcionalidad. El itinerario que se recomienda es el
siguiente. Si tiene dudas sobre cómo realizar alguna de las acciones descritas,
revise el~\nameref{chap:manual_usuario}.

El primer paso antes de nada es lanzar la batería de tests que viene incluida en
el proyecto, mediante el comando

\begin{verbatim}
    $ python tests.py test
\end{verbatim}

Todos los tests deberían funcionar correctamente. Si no lo hacen, es que ha
habido un fallo en el sistema o en la aplicación.

Tras esto, el código clonado dispone de varios ejemplos implementados usando la herramienta,
por lo que bastaría con ir a a la carpeta \texttt{examples} y ejecutar sitic en cualquier
de los ejemplos disponibles. Los comandos para este caso serían los siguientes:

\begin{bashcode}
    $ workon sitic
    $ cd ~/sitic/examples/blog
    $ sitic
\end{bashcode}

En el caso de que la generación del sitio de ejemplo no se llevara a cabo
correctamente, es que no se habrá realizado adecuadamente alguno de los pasos de instalación.
