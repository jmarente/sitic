\section{Descripción}

Sitic es un framework para hacer sitios web de uso general. Técnicamente hablando, Sitic es un generador de sitios web estáticos.
A diferencia de otros sistemas que genera dinámicamente una página cada vez que un visitante la solicita, Sitic crea el sitio
un única vez, 
cuando creas su contenido. Dado que los sitios web se ven con mucha más frecuencia de lo que se edita,

Los sitios construidos con Sitic son bastante más rápidos y seguros que un sitio generado de forma dinámica.
Se pueden alojar en cualquier lugar, incluyendo GitHub Pages, Google Cloud Storage o Amazon
S3 entre otros. Los sitios de Sitic se ejecutan sin depender de tiempos de ejecución costosos como Ruby, Python
o PHP y sin dependencia de ninguna base de datos.

\subsection{Diferencia con los generadores dinámicos}

Los generadores de sitios web generan contenidos en ficheros HTML. La mayoría son "generadores dinámicos".
Esto significa que el servidor HTTP (que es el programa que se ejecuta en su sitio web con el que el navegador del
usuario habla) ejecuta el generador para crear un nuevo fichero HTML cada vez que un usuario desea ver una página.

Crear la página de forma dinámica significa que la máquina que aloja el servidor HTTP tiene que tener suficiente
memoria y CPU para ejecutar el generador de forma eficaz durante todo el día. Si no, entonces el usuario tiene que
esperar a que la página se genere.

Nadie quiere que los usuarios esperen más de lo necesario, por lo que los generadores de sitios dinámicos programaron
sus sistemas para almacenar en caché los ficheros HTML. Cuando un fichero se almacena en caché, una copia se
almacena temporalmente en el equipo. Es mucho más rápido que el servidor envíe esa copia la próxima vez que
se solicite la página en lugar generarla desde cero.

Sitic intenta llevar el almacenamiento en caché un paso más allá. Todos los ficheros HTML se representan en su máquina.
Puede revisar los ficheros antes de copiarlos en la máquina que aloja el servidor HTTP. Dado que los ficheros HTML
no se generan dinámicamente, decimos que Sitic es un "generador estático".

No tener que ejecutar la generación de HTML cada vez que se recibe una petición tiene varias ventajas. Entre ellas,
la más notable es el rendimiento, los servidores HTTP son muy buenos en el envío de ficheros. Tan bueno que puede
servir eficazmente el mismo número de páginas con una fracción de memoria y CPU necesaria para un sitio dinámico.

Sitic tiene dos componentes para ayudarle a construir y probar su sitio web. El que probablemente usará más a menudo es el
servidor HTTP incorporado. Cuando ejecuta el servidor, Sitic procesa todo su contenido en ficheros HTML y luego ejecuta
un servidor HTTP en su máquina para que pueda ver cómo son las páginas.

El segundo componente se utiliza una vez que el sitio esté listo para ser publicado.
Ejecutar Sitic sin ninguna acción reconstruirá su sitio web completo utilizando la configuración \texttt{base\_url} 
del fichero de configuración de su sitio. Eso es necesario para que sus enlaces de página funcionen correctamente 
con la mayoría de las empresas de alojamiento.

\subsection{Características principales}

En términos técnicos, Sitic toma un directorio fuente de ficheros 
y plantillas y los usa como entrada para crear un sitio web completo.

Sitic cuenta con las siguientes características:

\paragraph{General}

\begin{itemize}
\item Tiempos de generación rápidos
\item Fácil instalación
\item Posibilidad de alojar su sitio en cualquier lugar
\end{itemize}

\paragraph{Organización}

\begin{itemize}
\item Organización sencilla
\item Soporte para secciones
\item URL personalizables
\item Soporte para taxonomías configurables que incluyen categorías y etiquetas.
\item Capacidad de clasificar el contenido como usted desea
\item Generación automática de tabla de contenido
\item Creación dinámica de menú
\item Soporte de URLs legibles
\end{itemize}

\paragraph{Contenido}

\begin{itemize}
\item Soporte nativo para contenido escrito en Markdown, Textile y Reestructured text
\item Soporte internacionalización
\item Soporte para los metadatos TOML y YAML en los contenidos
\item Páginas completamente personalizables
\end{itemize}
